\chapter{Branch Flow Model: Relaxations and Convexification}

In \cite{bfm01} the authors came up the Relaxed Branch Flow Model, and showed that in the case of Tree/Radial networks, the Relaxed Model can solve for the unique optimal solution, including the bus angles, and in the case of weakly meshed networks, there is a mechanism for extracting the bus angles from the relaxed solution, to find out its unique solution, if it exists.
%\newcommand{\fromto}[2]{$#1 \rightarrow #2$}

\newcommand{\fromto}[1]{%
	\StrBefore{#1}{,}[\from]%
	\StrBehind{#1}{,}[\to]%
	\from \rightarrow \to%
}

Legend for \cref{tab:bfm1}:
\begin{table}[h]
%	\caption{Notations}
	\caption{Table describing the variables involved in the Branch Flow Model equations.}
	\label{leg:bfm1}
	\begin{tabular}{cc}
		\toprule
		Symbol & Meaning \\
		\midrule
		$p_j, q_j$ & Real, Reactive Power flowing from bus $j$ into the network. \\
		$P_{ij}, Q_{ij}$ & Real, Reactive Power flowing in branch $(i, j)$ (sending-end). \\
		$I_{ij}, l_{ij}$ & Complex Current flowing in branch \\
		\bottomrule
	\end{tabular}
\end{table}

\begin{table}[h]
	\caption{Table describing the Branch Flow Model equations.}
	\label{tab:bfm1}
	\centering
	\hspace*{-2cm}
	%\documentclass{standalone}
%\usepackage{booktabs}

%\begin{document}
	\small
	\begin{tabular}{clccc} 
		\toprule
		Equation \# & Equation & Unknowns & Knowns & No. of Equations \\
		\midrule
		13 & $p_j = \Sigma P_{jk} + \Sigma (P_{ij} - r_{ij}l_{ij}) + g_jv_j$ & \begin{tabular}{c}
			$1 \times p_0$ \\
			$m \times P_{ij}$ \\
			$m \times l_{ij}$ \\
			$n \times v_j$
		\end{tabular} &
		\begin{tabular}{c}
			$n \times p_j$ \\
			$m \times r_{ij}$ \\
			$(n+1) \times g_{j}$ \\
			$1 \times v_0$
		\end{tabular} & $(n+1)$ \\
		\midrule
		14 & $q_j = \Sigma Q_{jk} + \Sigma (Q_{ij} - x_{ij}l_{ij}) + b_jv_j$ & \begin{tabular}{c}
			$1 \times q_0$ \\
			$m \times Q_{ij}$ \\
			$m \times l_{ij}$ \\
			$n \times v_j$
		\end{tabular} &
		\begin{tabular}{c}
			$n \times q_j$ \\
			$m \times x_{ij}$ \\
			$(n+1) \times b_{j}$ \\
			$1 \times v_0$
		\end{tabular} & $(n+1)$ \\
		\midrule
		15 & $v_j = v_{i} +  (r_{ij}^2 + x_{ij}^2)l_{ij} - 2(r_{ij}P_{ij} + x_{ij}Q_{ij})$ & \begin{tabular}{c}
			$m \times P_{ij}$ \\
			$m \times Q_{ij}$ \\
			$m \times l_{ij}$ \\
			$n \times v_j$
		\end{tabular} &
		\begin{tabular}{c}
			$b \times r_{ij}$ \\
			$m \times x_{ij}$ \\
			$1 \times v_0$
		\end{tabular} & $m$ \\
		\midrule
		16 & $l_{ij} = \frac{P_{ij}^2 + Q_{ij}^2}{v_j}$ & \begin{tabular}{c}
			$m \times P_{ij}$ \\
			$m \times Q_{ij}$ \\
			$m \times l_{ij}$ \\
			$n \times v_j$
		\end{tabular} &
		\begin{tabular}{c}
			$1 \times v_0$
		\end{tabular} & $m$ \\
		\midrule
		13 to 16 & {} & \begin{tabular}{c}
			$1 \times p_0$ \\
			$1 \times q_0$ \\
			$m \times P_{ij}$ \\
			$m \times Q_{ij}$ \\
			$m \times l_{ij}$ \\
			$n \times v_j$
		\end{tabular} &
		\begin{tabular}{c}
			$n \times p_j$ \\
			$n \times q_j$ \\
			$m \times r_{ij}$ \\
			$m \times x_{ij}$ \\
			$(n+1) \times g_j$ \\
			$(n+1) \times b_{j}$ \\
			$1 \times v_0$
		\end{tabular} & $2(n+1+m)$ \\
		\midrule
		{} & {} & $2(n+1+m)$ & $4n+2m+3$ & $2(n+1+m)$ \\
		\bottomrule
	\end{tabular}
%\end{document}

\end{table}

