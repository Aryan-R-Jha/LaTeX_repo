\chapter{Models: Battery Model for Multi-Period OPF}

\definecolor{darkgreen}{rgb}{0.0,0.5,0.0}

\newenvironment{colalign}[1]{%
    \color{#1}
    \align
}{%
    \endalign
}

All subscripts $j$ for a variable imply the node in the power grid (for node $j$).
All superscripts $t$ refer to the time period number $t$.

\begin{table}[htbp]
	\begin{threeparttable}
	\label{tab:grid_Nazir2018Jun}
	\centering
	\caption{Description of Grid Parameters}
		\begin{tabular}{>{\raggedright\arraybackslash $}p{2.5cm}<{$}
			>{\raggedright\arraybackslash}p{7.5cm}}
			\toprule
			\text{Variable} & \text{Description}                                                   \\
			\midrule
			\mathcal{N}     & {Set of all nodes. $\mathcal{N} = \{1,2, \ldots n\}$}                \\
			\mathcal{L}     & {Set of all branches. $\mathcal{L} =
			\{1,2, \ldots l\} = \{(i, k)\} \subset (\mathcal{N} \times \mathcal{N})$.}             \\
			\mathcal{D}     & {Set of all nodes containing DERs. $\mathcal{D}
			\subset \mathcal{N}$}                                                                  \\
			\mathcal{B}     & {Set of all nodes containing storage. $\mathcal{B}
			\subset \mathcal{N}$}                                                                  \\
			{\Delta t}      & {Duration of a single time period. Here $\Delta t = 15
			\text{ min.} = 0.25 \text{ h}$.}                                                       \\
			{T}               & {Prediction Horizon Duration. Total number of time-intervals solved for as part
			of one instance of MP-OPF.}                                                            \\
			\bottomrule
		\end{tabular}%
	\end{threeparttable}
\end{table}%

\begin{table}[htbp]
	\label{tab:bfm_variables}
	\centering
	\caption{Description of Branch Flow Model Variables}
	\begin{tabular}{>{\raggedright\arraybackslash $}p{2.5cm}<{$}
		>{\raggedright\arraybackslash}p{7.5cm}}
		\toprule
		\text{Variable} & \text{Description}                                                   \\
		\midrule
		{p_j^t}     & {Fixed Real Power Generation minus Fixed Real Power Load.
		Here, $p_j^t = p_D{_j}^t - p_L{_j}^t$. A known (predicted) value $\forall \, t, j$.}                \\
		{q_j^t}     & {Fixed Reactive Power Generation minus Fixed Reactive Power Load.
		Here, $q_j^t = - q_L{_j}^t$. A known (predicted) value $\forall \, t, j$.} \\
		{p_{D_j}^t} & {Real Power Generated by DERs} \\
		{p_{L_j}^t} & {Real Power Demand} \\
		{q_{L_j}^t} & {Reactive Power Demand} \\
		\bottomrule
	\end{tabular}%
\end{table}%

\begin{table}[htbp]
	\begin{threeparttable}
	\label{tab:bounds_bfm_Nazir2018Jun}
	\centering
	\caption{Values, Lower Bounds and Upper Bounds on BFM Variables}
	\begin{tabular}{>{\raggedright\arraybackslash $}p{2.5cm}<{$}
			>{\raggedright\arraybackslash $}p{5.5cm}<{$}
		>{\raggedright\arraybackslash}p{4.5cm}<{}}
		\toprule
		\text{Variable}   & \text{Value or Limits}                        & \text{Description}                    \\
		\midrule
		{P_{{DER}_{Max}}}         & {\sim[2,20] \, \text{kW}}     & {$P_{Rated}$ \text{ of corresponding DER.}}                                    \\
		{\left|q_{B_{Max}}\right|}     & {\sim[1,10] \, \text{kVAr}} & {$q_{D_{Rated}}$ \text{ of corresponding DER.}}                                    \\
		{V_{min}} & {0.95 \, pu} & {} \\
		{V_{Max}} & {1.05 \, pu} & {} \\
		{v} & {[V_{min}^2, V_{Max}^2]} & {Squared magnitude of nodal voltage} \\
		\bottomrule
	\end{tabular}%
\end{threeparttable}
\end{table}%

In \cite{Nazir2018Jun, Nazir2019Jun}, the batteries are modelled using
four state/control variables, which are:
\begin{table}[htbp]
	\label{tab:batt_Nazir2018Jun}
	\centering
	\caption{Description of Battery Variables}
	\begin{tabular}{>{\raggedright\arraybackslash $}p{2.5cm}<{$}
			>{\raggedright\arraybackslash}p{5cm}
			>{\centering\arraybackslash $}p{2.5cm}<{$}
		>{\centering\arraybackslash\arraybackslash $}p{2.5cm}<{$}}
			\toprule
		\text{Variable}                           & \text{Description}                                  & \text{Dimension} &
		\text{Dimension $pu$}                                                                                                        \\
			\midrule
		B_{n, t}                                  & State of Charge (SOC) of Battery after the
		$t$-th time interval.                   & [kWh]            & [pu\,h] \\
		P^c_{n, t}                                & Average Charging Power of the Battery during the
		$t$-th time interval.                     & [kW]                                                & [pu]                       \\
		P^d_{n, t}                                & Average Discharging Power of the Battery during the
		$t$-th time interval.                     & [kW]                                                & [pu]                       \\
		q_{B_{n, t}}                              & Average Reactive Power Output from the Battery
		Inverter during the $t$-th time interval. & [kVAr]                                              & [pu]                       \\
		\bottomrule
	\end{tabular}%
\end{table}%

\clearpage
\text{where, }
\begin{description}
	\item[$t \in \{1, 2, \ldots, T\}$] The index of the discretized time intervals, where $t$ represents
		the $t$-th time interval of duration $\Delta t$.
	\item[$n \in \mathcal{N}$] The node $n$ is an element of the set of all nodes
		in the power grid $\mathcal{N}$. Note that $n$ can be used both as an iterator or
		as the total number of nodes in the grid (i.e. the cardinality of $\mathcal{N}$),
		and its meaning should be obvious from context.
		% Add more items as needed
\end{description}

\begin{table}[htbp]
	\begin{threeparttable}
	\label{tab:bounds_batt_Nazir2018Jun_2}
	\centering
	\caption{Values, Lower Bounds and Upper Bounds on Battery Variables (Part 1/2)}
	\begin{tabular}{>{\raggedright\arraybackslash $}p{2.5cm}<{$}
			>{\raggedright\arraybackslash $}p{5.5cm}<{$}
		>{\raggedright\arraybackslash}p{4.5cm}<{}}
		\toprule
		\text{Variable}   & \text{Value or Limits}                        & \text{Description}                    \\
		\midrule
		{P_{Max}}         & {\sim[2,20] \, \text{kW}}     &  {$P_{Rated}$ \text{ of corresponding DER.}}                                   \\
		{\left|q_{B_{Max}}\right|}     & {\sim[1,10] \, \text{kVAr}} & {$q_{D_{Rated}}$ \text{ of corresponding DER.}}                                    \\
		{P_d, P_c}        & {[0, P_{Max}]}                                & {}                                    \\
		{q_B}        & {[-q_{B_{Max}}, q_{B_{Max}}]}                                & {Currently linearly modeled (is actually quadratic)}                                    \\
		{E_{Rated}}       & {P_{Max} \times 4\text{ h}}                   & {$4 \text{ h}$ of
		one-way Charging/Discharging at Maximum Power}                                                            \\
		{B}               & {[0.30E_{Rated}, 0.95E_{Rated}]}              & {$2.4 \text{ h}$ of
		one-way Charging/Discharging at Maximum Power}                                                            \\
		{B_0}             & {0.625E_{Rated}}                              & {Batteries start with an SOC value in
		the middle of their SOC range.}                                                                           \\
		{\eta_d , \eta_c} & {0.95}                                         & {}                                    \\
		\bottomrule
	\end{tabular}%
\end{threeparttable}
\end{table}%

\begin{table}[htbp]
	\begin{threeparttable}
	\label{tab:bounds_batt_Nazir2018Jun}
	\centering
	\caption{Values, Lower Bounds and Upper Bounds on Battery Variables (Part 2/2)}
	\begin{tabular}{>{\raggedright\arraybackslash $}p{2.5cm}<{$}
			>{\raggedright\arraybackslash $}p{5.5cm}<{$}
		>{\raggedright\arraybackslash}p{4.5cm}<{}}
		\toprule
		\text{Variable}   & \text{Value or Limits}                        & \text{Description}                    \\
		\midrule
		{\alpha} & {1\mathrm{e}{-3}} & {Coefficient of auxiliary objective function penalizing SCD. Value depends on the magnitude 
		of the loss term in the objective function} \\
		{\gamma} & {50} & {Coefficient of auxiliary objective function penalizing deviation of final SOC value from a reference.} \\
		\bottomrule
	\end{tabular}%
	\begin{tablenotes}
		\item[1] A note on $\alpha$:
		Too big a value of $\alpha$ would reduce both $P_c$ and $P_d$ terms to zero, whereas too small a value would not penalize SCD, causing physically infeasible solutions.
	\end{tablenotes}
\end{threeparttable}
\end{table}%
% 

\section*{Optimization Equations}

\subsection*{Original Problem: Mixed-Integer Nonlinear Optimization Model - Full Horizon}

\begin{gather}
    \min_{q_{D_j}^t,
	P_{c_j}^t, P_d{_j}^t, q_{B_j}^t} \quad
	\sum_{t = 1}^{T} \sum_{(i, j) \in \mathcal{L}} (r_{ij}l_{ij}^t) \\
	\begin{align}
		\text{s.t.} & {}\nonumber \\
		{p_j^t} & = {\sum_{(j, k) \in \mathcal{L}} P_{jk}^t - \sum_{(i, j) \in \mathcal{L}}\left\{P_{ij}^t - r_{ij}l_{ij}^t\right\} - P_{d_j}^t + P_{c_j}^t} && \\
		{q_j^t} & = {\sum_{(j, k) \in \mathcal{L}} Q_{jk}^t - \sum_{(i, j) \in \mathcal{L}}\left\{Q_{ij}^t - x_{ij}l_{ij}^t\right\} - q_{D_j}^t - q_{B_j}^t} && \\
		{v_j^t} & = {v_{i}^t +  \left\{r_{ij}^2 + x_{ij}^2\right\}l_{ij}^t - 2(r_{ij}P_{ij}^t + x_{ij}Q_{ij}^t)} && \\
		{l_{ij}^t} & = {\frac{(P_{ij}^{t})^2 + (Q_{ij}^{t})^2}{v_i^t}} \\
		{\color{darkgreen}{ B_{j}^{t} }} &\color{darkgreen}{=} { \color{darkgreen}{ B_{j}^{t-1} + z \Delta t  \eta_c P_{c_j}^t - (1-z) \Delta t\frac{1}{\eta_d} P_{d_j}^t } } \\
		{ B_{j}^{0} } &= { 0.5(soc_{max}+soc_{min})E_{Rated} = 0.625E_{Rated}} \\
		{ B_{j}^{T} } &= { B_{j}^{0}} \\
		{where,} & {} \\
		{(i, j)} &: {\text{Branch connecting nodes $i$ and $j$}} \\
		{p_j^t} &= {p_D{_j}^t - p_L{_j}^t} \\
		{q_j^t} &= {-q_L{_j}^t} \\
		{t} &= {\{1, 2, \ldots T\}} \\
		{z} &= {\{0, 1\}}
	\end{align}
\end{gather}

\subsection*{(Integer Constraint Relaxed) Naive Brute Force Full Optimization Model - Full Horizon}


\subsubsection*{Storage Modelling}

\begin{gather}
	\begin{align}
		{\color{black}{ B_{j}^{t} }} &\color{black}{=} { \color{black}{ B_{j}^{t-1} + \Delta t  \eta_c P_{c_j}^t - \Delta t\frac{1}{\eta_d} P_{d_j}^t } } \\
		{ B_{j}^{T} } &= { B_{j}^{0}} \\
		{ B_{j}^{0} } &= { 0.625E_{Rated, j} } \\
		{\text{where,}} & {} \nonumber\\
		{P^t_{c,j}} &: {\text{Average Charging Power of Battery $j$ during time-step $t$}} \nonumber\\
		{P^t_{d,j}} &: {\text{Average Discharging Power of Battery $j$ during time-step $t$}} \nonumber\\
		{B^t_j} &: {\text{SOC of Battery $j$ at the end of $t$ time-steps}} \nonumber\\
		{q^t_{Bj}} &: {\text{Average Reactive Power of Battery $j$ during time-step $t$ }} \nonumber\\
		{B^0_j} &: {\text{SOC of Battery $j$ at the beginning of the horizon}} \nonumber\\
		{P_{Rated, j}} &: {\text{Rated Charging/Discharging Power of Battery $j$}} \nonumber\\
		{E_{Rated, j}} &: {\text{Rated Maximum SOC of Battery $j$}} \nonumber\\
		{S_{Rated, j}} &: {\text{Rated Apparent Power of Inverter at Battery $j$}} \nonumber\\
		{ soc_{min}, soc_{max} } &= {0.30, 0.95} \\
		{ \eta_c, \eta_d } &= {0.95, 0.95} \\
		{P^t_{cj}, P^t_{dj}} &\in {[0, P_{Rated, j}]} \\
		{S_{Rated,j}} &= {1.2P_{Rated,j}} \\
		{ (P^t_{dj} - P^t_{cj})^2 + (q^t_{Bj})^2} &\leq {(S_{Rated,j})^2} \\
		% {B^t_{j}} &\in {[soc_{min}E_{Rated, j}, soc_{max}E_{Rated, j}]} \\
		{B^t_{j}} &\in {[soc_{min}, soc_{max}]*E_{Rated, j}} \\
		{q^t_{Bj}} &\in {[-\sqrt{0.44}, \sqrt{0.44}]*S_{Rated,j}} \\
		{t} &= {\{1, 2, \ldots T\}}
	\end{align}
\end{gather}

\begin{gather}
    % \min_{P_{ij}^t, Q_{ij}^t, v_{j}^t, l_{ij}^t, q_{D_j}^t, B_{j}^{t},
	% P_{c_j}^t, P_d{_j}^t, q_{B_j}^t} \quad
	% \sum_{t = 1}^{T} \sum_{(i, j) \in \mathcal{L}} (r_{ij}l_{ij}^t) + 
	% \alpha \sum_{t = 1}^{T} \sum_{j \in \mathcal{B}} \left\{ (1- \eta_c)P_{c_j}^t + \left(\frac{1}{\eta_d}-1\right) P_{d_j}^t \right\}
	% \\
	% + \gamma \sum_{j \in \mathcal{B}} \left\{ B^{T}_{j} - B_{ref}\right\}\\
	\begin{align}
		\min_{
		\substack{
		q_{D_j}^t, P_{c_j}^t, P_d{_j}^t, q_{B_j}^t}
		} 
		\vspace{-15ex} % Adjust the space here
		& \quad
		\sum_{t = 1}^{T} \sum_{(i, j) \in \mathcal{L}} (r_{ij}l_{ij}^t) \\
		&+ \alpha \sum_{t = 1}^{T} \sum_{j \in \mathcal{B}} \left\{ (1- \eta_c)P_{c_j}^t + \left(\frac{1}{\eta_d}-1\right) P_{d_j}^t \right\} \\
		&+ \gamma \sum_{j \in \mathcal{B}} \left\{ \left(B^{T}_{j} - B_{ref_{j}}]\right)^2\right\} \\
	% \end{align}
	% \begin{align}
		\text{s.t.} & {}\nonumber \\
		{p_j^t} & = {\sum_{(j, k) \in \mathcal{L}} P_{jk}^t - \sum_{(i, j) \in \mathcal{L}}\left\{P_{ij}^t - r_{ij}l_{ij}^t\right\} - P_{d_j}^t + P_{c_j}^t} && \\
		{q_j^t} & = {\sum_{(j, k) \in \mathcal{L}} Q_{jk}^t - \sum_{(i, j) \in \mathcal{L}}\left\{Q_{ij}^t - x_{ij}l_{ij}^t\right\} - q_{D_j}^t - q_{B_j}^t} && \\
		{v_j^t} & = {v_{i}^t +  \left\{r_{ij}^2 + x_{ij}^2\right\}l_{ij}^t - 2(r_{ij}P_{ij}^t + x_{ij}Q_{ij}^t)} && \\
		{l_{ij}^t} & = {\frac{(P_{ij}^{t})^2 + (Q_{ij}^{t})^2}{v_i^t}} \\
		{\color{darkgreen}{ B_{j}^{t} }} &\color{darkgreen}{=} { \color{darkgreen}{ B_{j}^{t-1} + \Delta t  \eta_c P_{c_j}^t - \Delta t\frac{1}{\eta_d} P_{d_j}^t } } \\
		{ B_{j}^{0} } &= { 0.5(soc_{max}+soc_{min})E_{Rated} = 0.625E_{Rated}} \\
		% { B_{j}^{T} } &= { B_{j}^{0}} \\
		{where,} & {} \\
		{(i, j)} &: {\text{Branch connecting nodes $i$ and $j$}} \\
		{p_j^t} &= {p_D{_j}^t - p_L{_j}^t} \\
		{q_j^t} &= {-q_L{_j}^t} \\
		{t} &= {\{1, 2, \ldots T\}}
	\end{align}
\end{gather}

\section*{Previous Simulations}
\subsection*{Step 2: Full Optimization Model - Single Time Step Greedy Approach}


\begin{gather}
    \min_{q_{D_j}^t, 
	P_{c_j}^t, P_d{_j}^t, q_{B_j}^t} \quad
	\sum_{(i, j) \in \mathcal{L}} (r_{ij}l_{ij}^t) + 
	\alpha \sum_{j \in \mathcal{B}} \left\{ (1- \eta_c)P_{c_j}^t + \left(\frac{1}{\eta_d}-1\right) P_{d_j}^t \right\} \\
	\begin{align}
		\text{s.t.} & {}\nonumber \\
		{p_j^t} & = {\sum_{(j, k) \in \mathcal{L}} P_{jk}^t - \sum_{(i, j) \in \mathcal{L}}\left\{P_{ij}^t - r_{ij}l_{ij}^t\right\} - P_{d_j}^t + P_{c_j}^t} && \\
		{q_j^t} & = {\sum_{(j, k) \in \mathcal{L}} Q_{jk}^t - \sum_{(i, j) \in \mathcal{L}}\left\{Q_{ij}^t - x_{ij}l_{ij}^t\right\} - q_{D_j}^t - q_{B_j}^t} && \\
		{v_j^t} & = {v_{i}^t +  \left\{r_{ij}^2 + x_{ij}^2\right\}l_{ij}^t - 2(r_{ij}P_{ij}^t + x_{ij}Q_{ij}^t)} && \\
		{l_{ij}^t} & = {\frac{(P_{ij}^{t})^2 + (Q_{ij}^{t})^2}{v_i^t}} \\
		{B_{j}^{t}} &= {B_{j}^{t-1} + \Delta t \eta_c P_{c_j}^t - \Delta t\frac{1}{\eta_d} P_{d_j}^t} \\
		{ B_{j}^{0} } &= { 0.5(soc_{max}+soc_{min})E_{Rated} = 0.625E_{Rated}} \\
		{where,} & {} \\
		{(i, j)} &: {\text{Branch connecting nodes $i$ and $j$}} \\
		{p_j^t} &= {p_D{_j}^t - p_L{_j}^t} \\
		{q_j^t} &= {-q_L{_j}^t} \\
		{t} &= {\{1, 2, \ldots T\}}
	\end{align}
\end{gather}

\subsection*{Step 1b: Initialisation Lossless Optimization Model WITH Batteries - Single Time Step Greedy Approach}


\begin{gather}
    \min_{q_{D_j}^t,
	P_{c_j}^t, P_d{_j}^t, q_{B_j}^t} \quad
	\sum_{(i, j) \in \mathcal{L}} \left\{r_{ij}\frac{(P_{ij}^{t})^2 + (Q_{ij}^{t})^2}{v_{0i}^{t}} \right\} + 
	\alpha \sum_{j \in \mathcal{B}} \left\{(1- \eta_c)P_{c_j}^t + \left( \frac{1}{\eta_d}-1 \right) P_{d_j}^t \right\} \\
	\begin{align}
		\text{s.t.} & {}\nonumber \\
		{p_j^t} & = {\sum_{(j, k) \in \mathcal{L}} P_{jk}^t - \sum_{(i, j) \in \mathcal{L}}\left(P_{ij}^t\right) - P_{d_j}^t + P_{c_j}^t} && \\
		{q_j^t} & = {\sum_{(j, k) \in \mathcal{L}} Q_{jk}^t - \sum_{(i, j) \in \mathcal{L}}\left(Q_{ij}^t\right) - q_{D_j}^t - q_{B_j}^t} && \\
		{v_{0j}^t} & = {v_{0i}^t - 2(r_{ij}P_{ij}^t + x_{ij}Q_{ij}^t)} && \\
		{B_{j}^{t}} &= {B_{j}^{t-1} + \Delta t  \eta_c P_{c_j}^t - \Delta t\frac{1}{\eta_d} P_{d_j}^t} \\
		{ B_{j}^{0} } &= { 0.5(soc_{max}+soc_{min})E_{Rated} = 0.625E_{Rated}} \\
		{where,} & {} \\
		{(i, j)} &: {\text{Branch connecting nodes $i$ and $j$}} \\
		{p_j^t} &= {p_D{_j}^t - p_L{_j}^t} \\
		{q_j^t} &= {-q_L{_j}^t} \\
		{t} &= {\{1, 2, \ldots T\}}
	\end{align}
\end{gather}

\subsection*{Step 1a: Initialisation Lossless Optimization Model WITHOUT Batteries - Single Time Step Greedy Approach}


\begin{gather}
    \min_{P_{ij}^t, Q_{ij}^t, v_{j}^t,  q_{D_j}^t} \quad 0 \\
	\begin{align}
		\text{s.t.} & {}\nonumber \\
		{p_j^t} & = {\sum_{(j, k) \in \mathcal{L}} P_{jk}^t - \sum_{(i, j) \in \mathcal{L}} P_{ij}^t } && \\
		{q_j^t} & = {\sum_{(j, k) \in \mathcal{L}} Q_{jk}^t - \sum_{(i, j) \in \mathcal{L}} Q_{ij}^t } && \\
		{v_j^t} & = {v_{i}^t - 2(r_{ij}P_{ij}^t + x_{ij}Q_{ij}^t)} && \\
		{where,} & {} \\
		{(i, j)} &: {\text{Branch connecting nodes $i$ and $j$}} \\
		{p_j^t} &= {p_D{_j}^t - p_L{_j}^t} \\
		{q_j^t} &= {-q_L{_j}^t} \\
		{t} &= {\{1, 2, \ldots T\}}
	\end{align}
\end{gather}

A simple metric called $P_{Save}$ gives an indication of the effect of power
generated by batteries and DERs which offset substation power 
(indirectly flowing it via its parent area if it is not directly connected to 
the substation).

Its formula is as shown:

\begin{align}
	{P_{Save}} &= {100\% * \left(\frac{\sum_{j \in \mathcal{B}}\left( P_{d_j} - P_{c_j} \right)}{P_{12} + 
	\sum_{j \in \mathcal{D}} P_{DER_j} +  \sum_{j \in \mathcal{B}}\left( P_{d_j} - P_{c_j} \right)} \right)}
\end{align}


Here are some key definitions used in this thesis, which have been aligned as per NERC's/NREL's technical documents \cite*{nercRes01,nercIBR00}:

\begin{itemize}
	\item Inverted-Based Resources (IBR) \cite*{nercRes01}
	\item Grid Following Interters (GFL): Depend on an external constant voltage source to generate current \cite*{nercIBR00}. They \textit{track} an AC Voltage Waveform
	\item Grid Forming Inverters (GFM): Generate their own constant voltage \cite*{nercIBR00}. They \textit{generate} their own AC Voltage waveform. Actually have been in use in off-grid power systems for decades. NERC Inverter-based Resource Performance Working Group (IRPWG) proposed a unified definition: ``An inverter that maintains a constant voltage phasor in
	the transient and sub-transient time frames"
	\item Short Circuit Ratio (SCR): (for a bus on the grid, potentially where an IBR is to be connected) Ratio of the total fault current (power) at the bus to the rated current (power) of the generator to be connected at the point. An indicator for the `grid strength' at the point. An SCR value greated that $5$ is considered healthy, and a value of less than $3$ is considered weak. For a weak point with low SCR, EMT studies are usually conducted by the grid operators to detect transient issues which may not be caught by conventional Dyanmic Studies.
	\begin{align*}
		SCR_{POI} &= \frac{SCMVA_{POI}}{MW_{VER}}
	\end{align*}
	Here, VER denotes a Variable Energy Resources, like a PV.
	\item Weighted Short Circuit Ratio (WSCR): (for a grid with many buses where IBRs are potentially to be connected) A weighted average of the SCRs across the whole grid, to denote a relative `grid strength'.
	\begin{align*}
		WSCR &= \frac{\Sigma^{N}_{j} SCMVA_{j} * P_{Rated MW j}}{(\Sigma^{N}_{j} P_{Rated MW j})^2}
	\end{align*}
\end{itemize}