\section{A Comparison of Select Correlation Coefficients}

\subsection*{Pearson's Correlation Coefficient}

Pearson's Correlation Coefficient \cite{pearsonCorrCoeff}, often denoted using the parameter $\rho$, is one of the go-to methods for computing the degree of correspondence between two datasets or time-series. The value of $\rho$ varies between positive and negative unity, i.e. $-1 \leq \rho \leq +1$, where the more the value is skewed towards either extreme end, the more correlated the two datasets or time-series are. The sign determines if the correlation is positive or negative. A value of $\rho$ closer to $0$ implies little to no correlation between the two variables. The statistic, like most other correlation coefficients, is often employed to test for hypothesis testing for any significant relationship between two dataset variables.

\Cref{tab:pearsonCorrCoeff01} gives a quick overview of this correlation coefficient.

\begin{table}[!ht]
	\centering
	\caption{Interpretation of different values of the Pearson's $\rho$ Correlation Coefficient.}
	\label{tab:pearsonCorrCoeff01}
	
	\begin{tabular}{ccc}
		\toprule
		\begin{tabular}{l}
			Pearson's Correlation\\
			Coefficient $\rho$
		\end{tabular} & Interpretation & \begin{tabular}{l}
			Examples of two variables \\
			specific to Power Systems \\
			following the relationship
		\end{tabular}\\ 
		\midrule
		$\rho \in (0, 1]$ & 
		\begin{tabular}{l}
			A positive value of \\
			the correlation coefficient\\
			implies that both the\\
			variables change in the\\
			same `direction'.
		\end{tabular} & 
		\begin{tabular}{l}
				\begin{tabular}{l}
					Population of a city \\
					\begin{tabular}{c}
						vs
					\end{tabular}\\
					Daily Power Demand
				\end{tabular}\\
				\midrule
				\begin{tabular}{l}
					Transmission Line Length\\
					\begin{tabular}{c}
						vs
					\end{tabular}\\
					Line Losses
				\end{tabular}
		\end{tabular}\\
		\midrule
		 $\rho \approx 0$ & 
		 \begin{tabular}{l}
		 	A near zero value of \\
		 	the correlation coefficient\\
		 	implies that both the\\
		 	variables are\\
		 	uncorrelated. In\\
		 	some special cases,\\
		 	such as the case of\\
		 	jointly Gaussian\\
		 	variables, a zero\\
		 	correlation implies the\\
		 	independence of the two\\
		 	variables. \cite{zeroCorrelationImpliesIndependenceForGaussianDistribution}
		 \end{tabular} & 
		 \begin{tabular}{l}
		 	\begin{tabular}{l}
		 		Nominal Frequency of Grid\\
		 		\begin{tabular}{c}
		 			vs
		 		\end{tabular}\\
		 		Wattage of end-user electronics
		 	\end{tabular}\\
		 	\midrule
		 	\begin{tabular}{l}
		 		Nominal Frequency \\
		 		\begin{tabular}{c}
		 			vs
		 		\end{tabular}\\
		 		Setting of a Distance Relay
		 	\end{tabular}
		 \end{tabular}\\
	 \midrule
	 $\rho \in [-1, 0)$ & 
	 \begin{tabular}{l}
	 	A negative value of \\
	 	the correlation coefficient\\
	 	implies that the\\
	 	variables change in\\
	 	opposite `directions'.
	 \end{tabular} & 
	 \begin{tabular}{l}
	 	\begin{tabular}{l}
	 		Critical Clearing angle for a fault\\
	 		\begin{tabular}{c}
	 			vs
	 		\end{tabular}\\
	 		Maximum Loadability of a Line
	 	\end{tabular}\\
	 	\midrule
	 	\begin{tabular}{l}
	 		Bus Voltage\\
	 		\begin{tabular}{c}
	 			vs
	 		\end{tabular}\\
 			Line current flowing \\
 			into that Bus
	 	\end{tabular}
	 \end{tabular}\\
 	\bottomrule
	\end{tabular}
\end{table}

\begin{figure}[!ht]
%	\captionsetup[subfloat]{farskip=2pt,captionskip=1pt}
	\centering
	\subfloat[Weak Positive Correlation]{\scalebox{0.40}{\import{../figures/}{pearsonCC01.pdf_tex}}}
	\hspace{-20pt}
	\subfloat[Strong Positive Correlation]{\scalebox{0.40}{\import{../figures/}{pearsonCC02.pdf_tex}}}
	
	\subfloat[Weak Negative Correlation]{\hspace{35pt}\scalebox{0.40}{\import{../figures/}{pearsonCC03.pdf_tex}}}
	\hspace{-10pt}
	\subfloat[Strong Negative Correlation]{\scalebox{0.40}{\import{../figures/}{pearsonCC04.pdf_tex}}}
	
	\subfloat[Near Zero Correlation]{\scalebox{0.40}{\import{../figures/}{pearsonCC05.pdf_tex}}}
	
	\caption{Pearson's Correlation Coefficient for different plots of two variables}
	
	\label{fig:pearsonCC}
\end{figure}

\subsection*{Spearman's Ranked Correlation Coefficient}

\subsection*{Kendall's Correlation Coefficient}

\subsection*{Modified Kendall's Correlation Coefficient}