\section[Introduction]{Introduction}
\label{sec:intro}

Unlike transient faults in a power grid which can generally be attributed to a sudden but tangible anomaly (corrective outcomes of protection mechanisms, sudden failure of a generator or transformer, line faults), certain indicators of proximity to instabilities may remain undetected until they accumulate over time to manifest as a major upset to the grid \cite{entsoeReportGridCollapseContinentalEurope2021Jan} and/or make the grid less robust/more susceptible to collapses \cite{schafer01}. The causes of these disturbances is often an accruing of stochastic perturbations in the state variables of the grids when it is already stressed by an increased power demand \cite{rosehartBifurcationAnalysisOfVariousPowerSystemModels}.  The accrual of stochastic perturbations in turn can be caused due to various physical phenomena, including measurement errors, distributed renewable generation fluctuations \cite{adeen01}, sudden gaps in power demand and supply due to power trading \cite{schafer01}, strict operational frequency deadbands \cite{vorobev01, francesca01}, etc.
 
While modeling every significant possible source of stochastic disturbance can be difficult or perhaps even outright impossible, at least their detection can be made through model free data-driven statistical analysis, enabling early detection of grid stability problems for a timely course-corrective action \cite{schafer01, sanchez01, ghanvati01}.

Bifurcation Theory \cite{nathanKutzNotesOnBifurcationTheoryAndNormalForms, rosehartBifurcationAnalysisOfVariousPowerSystemModels, chenBifurcationsAndChaosInEngineering, mohlerDyanmicsAndControlPartOne} helps explain the erratic functioning of stressed dynamical systems such as the power grid, and the theory of Critical Slowing Down \cite{schefferEarlyWarningSignalsForCriticalTransitions} lists tangible quantitative analysis tools which can help us detect an impending `bifurcation' (blackout) in the power grid.

In this thesis, firstly various real-world grid frequency time series archives were investigated on their robustness against minor disturbances and any kind of long-standing stability problems in them, through the use of bulk distribution probability density functions and autocorrelation decay plots. This analysis is referred to as Offline/Postmortem analysis as the entire span of input data used has already been sampled before the analysis for a long period (several months or years). The chapter concludes with how different grids have varying levels of susceptibility to instabilities based on their size and their inherent dynamics as well as control and operation mechanisms. 

Next, the effectiveness of two statistical parameters computed in real-time listed out as per the theory of Critical Slowing Down, namely Autocorrelation and Variance as Early Warning Sign Indicators of an approaching bifurcation in a power grid. For statistically testing for the increase in autocorrelation and variance of the time-series, a new kind of statistical parameter, namely the Modified Kendall's Tau Correlation Coefficient was introduced, to accommodate for the confidence margins (p-values) associated with the computed correlation coefficients. This analysis has been labeled as the Online/Real-time analysis as the input data here is only in instantaneously available from a stream of data, such as PMU data. It was found that both Fixed Lag Autocorrelation and Variance of the bus voltages serve as reliable early warning signs of indicators to instabilities in power grids. Through the Modified Kendall's Tau Correlation Coefficients, it was visually observed how different buses may be more or less susceptible to a uniform grid-level stress (in the form of a steadily increasing power demand).

The rest of the thesis is as below:
Chapter \ref{sec:litt}: Literature Review and Objectives provides and briefly explains the relevant work performed by the scientific community on predicting early warning signs for instability indicators, followed by how the work done in this thesis provides original contributions to the existing body of literature.  \\
Chapter \ref{sec:theory}: Theory provides the mathematical foundation required for justifying the Offline Analysis in Chapter \ref{sec:offline} and  Online Analysis in Chapter \ref{sec:online}.\\
Chapter \ref{sec:offline}: Offline Analysis provides two kinds of methods of statistically analyzing the bulk distribution of a state variable (grid frequency) of the grid and how meaningful inferences could be made from them. This topic is also called as the Post-mortem analysis because the analysis is performed on data acquired over a long duration of operation of various real-world grids.\\
Chapter \ref{sec:online}: Online Analysis showcases a simulation which uses all of the concepts described in the Theory section in order to test an algorithm for detect early warning signs for indicators to instability. This topic is also called as real-time analysis as the methods described in the chapter are supposed to be done over a real-time stream of data, such as data obtained from a PMU.\\
Chapter \ref{sec:concl} and Chapter \ref{sec:future} are for Conclusions and Future Work respectively.
