\section{Problem 4}
Consider an experiment that has four outcomes, $A$, $B$, $C$, and $\square$.

\begin{enumerate}[a.]
	\item How many events can be defined for this experiment? What are they?
	\item  What is the sample space for the experiment?
	\item For this part, assume that $P(\{A, B, \square\}) = 0.8$, $P(\{A, C\}) = 0.4$ and $P(\{A, \square\}) = 0.4$.\\
	Please find the probabilities of all events defined for this experiment.
	\item For this part, assume that the probabilities $P(\{A, B, \square\})$ and $P(\{A, C\})$ are known, and that we can measure the probability of exactly one more event. Please find all other events such that the knowledge of the three events' probabilities allows us to compute the probabilities of all events.
\end{enumerate}

\subsection{Solution}
Let the event be called $E_4$.\\
Given $\Omega$ = $\set{A, B, C, \square}$
\begin{enumerate}[a.]
	\item \# Events = $2^4 = 16$. They are the elements of the powerset $\powerset$ made from $\Omega$\\
	\begin{align*}
		\powerset &= \{ \set{\phi}, \set{A} \cdots \set{\square}, \nonumber\\
		&{} \set{A, B}, \cdots \set{C, \square}, \nonumber\\
		&{} \set{A, B, C}, \cdots \set{B, C, \square}, \nonumber \\
		&{} \set{A, B, C, \square}\} 
	\end{align*}
	Refer to \cref{tab:probFourOutcomes} for the full list.
	\item Sample Space $\Omega$ = $\set{A, B, C, \square}$
	\item Since $A$, $B$, $C$, and $\square$ are all outcomes, they are disjoint events. Axiom 3 may be used to compute the probabilities of their unions.
		\begin{align*}
			P(\set{A, B, \square}) &= 0.8 \nonumber\\
			\implies P(C) &= 0.2 && \text{(Axiom 2)} \nonumber\\
			P(\set{A, C}) &= 0.4 \nonumber\\
			\implies P(A) &= 0.2 \nonumber \\
			P(\set{A, \square}) &= 0.4 \nonumber \\
			\implies P(\square) &= 0.2 \nonumber \\
			\implies P(B) &= 0.4 \nonumber		 
		\end{align*}
		
		Refer to \cref{tab:probFourOutcomes} for the event-wise probabilities.
		\begin{table}[!htpb]
			\centering
			\caption{Probabilities of all events of Experiment $E_4$. The `Sufficient?' column is meant for part d. of the problem where knowledge of the event's probability value when combined with two other given probability values is sufficient for determining the probabilities of all possible events of $E_4$.}
			\label{tab:probFourOutcomes}
			\begin{tabular}{LLLl}
				\toprule
				S. No. & Event & Probability & Sufficient?\\
				\midrule
				1. & \phi & 0  & No\\
				2. & A & 0.2 & No\\
				3. & B & 0.4 & Yes\\
				4. & C & 0.2 & No \\
				5. & \square & 0.2 & Yes\\
				6. & \set{A, B} & 0.6 & Yes\\
				7. & \set{A, C} & 0.4 & No\\
				8. & \set{A, \square} & 0.4 & Yes\\
				9. & \set{B, C} & 0.6 & Yes\\
				10. & \set{B, \square} & 0.6 & No\\
				11. & \set{C, \square} & 0.4 & Yes\\
				12. & \set{A, B, C} & 0.8 & Yes\\
				13. & \set{A, B, \square} &0.8 & No\\
				14. & \set{A, C, \square} & 0.6 & Yes\\
				15. & \set{B, C, \square} & 0.8 & No\\
				16. & \set{A, B, C, \square} & 1.0 & No\\
				\bottomrule
			\end{tabular}
		\end{table}
	\item On similar lines as the previous part c. of this problem, since we are able to find out the values of $P(A)$ and $P(C)$ from the given values, we are only interested in knowing the values of either $P(B)$ or $P(\square)$. Only events which are supersets of unions of both $B$ and $\square$ outcomes are unhelpful in determining their individual values. Refer to the `Sufficient?' column in \cref{tab:probFourOutcomes}.
\end{enumerate}

\noindent\rule{\textwidth}{1pt}