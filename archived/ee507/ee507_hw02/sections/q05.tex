\section{Problem 5}
\begin{enumerate}[5a.]
	\item Please define a random variable, and explain why the concept is important.
	\item Please define the notion of a cumulative distribution function (CDF).
\end{enumerate}

\subsection{Solution}


\begin{enumerate}[5a.]
	\item A Random Variable is an injective (one-one) mapping from the outcomes of an experiment to countable numbers. Random Variables are a good choice to classify outcomes due to the following reasons:
		\begin{itemize}
			\item Experiments which have a continuous, and therefore infinite, set of outcomes, can only be represented by random variables, and not nominal names, practically speaking.
			\item Random Variables can be used to efficiently trim the set of all possible outcomes into manageable bins of events.
			\item Random Variables often carry more meaning than a nominal name for an outcome in most physical/simulated experiments. For example to measure the height of a population, it makes sense to assign every possible height (with an appropriate level of quantization) as an outcome itself. If two random variables appear `close' in value, then the two variables may also be conveying outcomes in the physical/simulated world which are also close to each other. For example $H=164cm$ and $H=165cm$ as random variables for describing the distribution in an adult human population are outcomes which can be considered in proximity to one another (i.e. both the outcomes classify adults of almost similar heights), whereas they are `far-apart' from $H=190cm$, both in random variable value and in physical meaning of the outcome they are representing.
		\end{itemize} 
	\item CDF$_x(\alpha) = F_X(\alpha) = P(x \leq \alpha)$ for a random variable $X$ is a convenient way to represent the probability of the set of all outcomes which are below a certain threshold $\alpha$. This definition can then be extrapolated to query for random variables in a specific range, like $P_X(\alpha_1 < x < \alpha_2)$\\
	Taking the same example as above of an experiment of sampling an adult human and measuring their height, it does not make sense to seek humans who have an exact height of, say $161.327cm$. Rather, the experimenter is usually interested in range queries, such as the probability that a randomly selected adult human has a height between $160cm$ and $165cm$, in which case the value can be conveniently calculated by computing the CDFs at the two heights and taking their difference. None the less, an exact pdf can always be computed by taking the derivative of the CDF wrt the random variable at the desired value.
\end{enumerate}

\noindent\rule{\textwidth}{1pt}