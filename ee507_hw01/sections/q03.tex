\section{Problem 3}
Consider any uncertain experiment, and let $A$, $B$, and $C$ be three events defined for the experiment. Please prove the following equalities or inequalities. In doing so, you may use the three axioms, any equalities/inequalities developed in class, standard set theory and algebra concepts, and earlier parts of the problem.

\begin{enumerate}[a.]
	\item
	\!  %one-sixth of a quad's space, the Stack Exchange answer did this so I'm doing the same too.
	$
	\begin{aligned}
		P(A\bar B) = P(A) -  P(AB) \nonumber
	\end{aligned}
	$
	\item 
	\! If $A$ is contained in $B$, then $P(A) \leq P(B)$.
	\item  
	\! 
	$
		\begin{aligned}
			P(A+B) + P(\bar A) + P(\bar B) - P(\bar A + \bar B) = 1 \nonumber
		\end{aligned}
	$ 
	\item
	\!  
	$
		\begin{aligned}
			P(AB) + P(\bar A C) + P(\bar B \bar C) \leq 1 \nonumber
		\end{aligned}
	$ \\[5pt]
	Please also give an example showing that the bound can be achieved.
\end{enumerate}

\subsection{Solution}

\begin{enumerate}[a.]
	\item To Prove: $P(A \bar{B}) = P(A) - P(AB)$
		\!
		\begin{proof}
			\begin{align}
				& \text{From the Law of Total Probability, we know that } \nonumber\\ 
				& \quad P(A) = P(A \bar{B}) + P(AB) \label{eq:totalProbLaw} \\
				& \text{Rearranging \cref{eq:totalProbLaw}, we get:} \nonumber\\
				& \quad P(A\bar{B}) = P(A) - P(AB) \\
				& \text{ Hence Proved. \Laughey[1.4]} \nonumber
			\end{align}
		\end{proof}
	\item To Prove: If $A$ is contained in $B$, then $P(A) \leq P(B)$
		\!
		\begin{proof}
			\begin{align}
				& \text{To Prove: } \text{if } A \subseteq B \text{,  then } P(A) \leq P(B) \nonumber \\
				& \text{From the Law of Total Probability, we know that} \nonumber\\
				& \quad P(B) = P(B\bar{A}) + P(BA) \label{eq:totalProbLawB}\\
				& \quad A \subseteq B \implies A \cap B = A \implies P(BA) = P(A) \nonumber\\
				& \quad \text{or,} \quad P(B) = P(B\bar{A}) + P(A) \nonumber\\
				& \text{But, by Axiom $1$ of Probability} \nonumber\\
				& \quad P(B\bar{A}) \geq 0 \nonumber\\
				& \quad \implies P(A) \leq P(B) \nonumber \\
				& \text{Hence Proved. \Laughey[1.4]} \nonumber			
			\end{align}
		\end{proof}
		\item \text{To Prove: } $P(A+B) + P(\bar A) + P(\bar B) - P(\bar A + \bar B) = 1$\\
		\!
			\begin{proof}
				\begin{align}
					LHS &= \{P(A) + P(B) - P(AB)\} + P(\bar{A}) + P(\bar{B}) &&\text{(Expanding the $LHS$)} \nonumber\\
					&- \{P(\bar{A}) + P(\bar{B}) - P(\bar{A}\bar{B})\}  \nonumber\\
					&= P(A) + P(B) - P(AB) + P(\bar{A}\bar{B}) \nonumber\\
					&= P(A) - P(AB) + P(\bar{A}\bar{B}) + P(B)&& \text{(Rearranging the equation)}\nonumber\\
					&= P(A\bar{B}) + P(\bar{A}\bar{B}) + P(B) &&\text{(Using \cref{eq:totalProbLaw})} \nonumber\\
					&= P(\bar{B}) + P(B) &&\text{(Using \cref{eq:totalProbLawB})} \nonumber\\
					&= 1 &&\text{Hence Proved \Laughey[1.4]}\nonumber
				\end{align}
			\end{proof}	
	\item To Prove: $P(AB) + P(\bar A C) + P(\bar B \bar C) \leq 1$ \\
	Also give an example for the equality case.
	\!
		\begin{proof}
			\begin{align}
				AB \cap \bar{A}C \cap \bar{B}\bar{C} &= \phi && \text{(Pairwise Disjoint Sets)} \nonumber\\
				\implies P(AB) + P(\bar{A}C) + P(\bar{B}\bar{C}) &= P( AB + \bar{A}C + \bar{B}\bar{C} ) && \text{(Axiom 3)} \nonumber\\
				&\leq 1 && \text{(Axiom 1)} \nonumber\\
				&{} && \text{Hence Proved \Laughey[1.4]} \nonumber				
			\end{align}
		\end{proof}
	
		For the special case of equality\\
		\begin{align}
			P(AB) + P(\bar{A}C) + P(\bar{B}\bar{C}) &= 1 \label{eq:specialCase}
		\end{align}
		We may equate the constituent events of \cref{eq:specialCase} to trivial identities in order to potentially reverse engineer the relationships between the three events $A$, $B$ and $C$:\\[5pt]
		Let\\[-20pt]
		\begin{align}
			AB + \bar{A}C + \bar{B}\bar{C} &= A + \bar{A}
		\end{align}
		Making comparisons, we get:
		\begin{align}
			AB &= A \nonumber\\
			\implies A &\subseteq B \label{eq:tmp1}\\
			\bar{A}C &= \bar{A} \nonumber\\
			\implies \bar{A} &\subseteq C \label{eq:tmp2}\\
			\bar{B}\bar{C} &= 0 \nonumber\\
			\implies \bar{B} \cap \bar{C} &= \phi \nonumber\\
			\implies B \cup C &= 1 \label{eq:tmp3}
		\end{align}
		
		\cref{eq:tmp2} gives two cases: Either $A$ and $C$ are disjoint events which exhaustively represent the sample space (i.e. $A$ $\cup$ $C$ = $\phi$ and $A+C$ = $\Omega$) or $C$ encompasses the whole sample space and $A$ is merely a subset of it (i.e. $C$ = $\Omega$ and $A \subseteq C$).
		
		One of the solutions satisfying \cref{eq:tmp1}, \cref{eq:tmp2} and \cref{eq:tmp3} is $A = B$ and $A \cap C = \phi$. \cref{fig:shape1} represents this instance of the relationship between $A, B$ and $C$.
		
		Another solution satisfying \cref{eq:tmp1}, \cref{eq:tmp2} and \cref{eq:tmp3} is $A \subseteq B \subseteq C$, represented by \cref{fig:shape2}.
		
		\begin{figure}[h]
			\centering
			\import{../figures/}{shape1Pdf.pdf_tex}
			\caption{Instance 1 which satisfies the equality criterion.}
			\label{fig:shape1}
		\end{figure}
		
		\begin{figure}[h]
			\centering
			\import{../figures/}{shape2Pdf.pdf_tex}
			\caption{Instance 2 which satisfies the equality criterion.}
			\label{fig:shape2}
		\end{figure}
	
		Let\\[-20pt]
		\begin{align}
			AB + \bar{A}C + \bar{B}\bar{C} &= C + \bar{C}
		\end{align}
		Making comparisons, we get:
		\begin{align}
			\bar{A}C &= C \nonumber\\
			\implies A &\subseteq \bar{C} \label{eq:tmp7}\\
			\bar{B}\bar{C} &= \bar{C} \nonumber\\
			\implies B &\subseteq C \label{eq:tmp8}\\
			AB &= 0 \nonumber\\
			\implies A \cap B &= \phi \label{eq:tmp9}
		\end{align}
		
		A solution satisfying \cref{eq:tmp7}, \cref{eq:tmp8} and \cref{eq:tmp9} is $A \subseteq B \subseteq C$, represented by \cref{fig:shape3}.
		
		\begin{figure}[h]
			\centering
			\import{../figures/}{shape3Pdf.pdf_tex}
			\caption{Instance 3 which satisfies the equality criterion.}
			\label{fig:shape3}
		\end{figure}
\end{enumerate}

\noindent\rule{\textwidth}{1pt}