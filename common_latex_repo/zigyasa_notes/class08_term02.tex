\documentclass{article}
\usepackage{amsmath,amssymb, amsthm}
\usepackage{cleveref}
\title{Notes for Zigyasa Ritwajeet Jha}
\date{}
\author{Aryan Ritwajeet Jha}

\begin{document}
\maketitle
%	\begin{align}
%		\intertext{Distance between Solwezi and Lusaka:}
%		distance &= 1500km \\
%		\intertext{Given, the relation between miles and kilometer is:}
%		mile &= 1.61km \\
%		distance\_in\_miles &= distance/miles
%		%LHS &= (2x+5y)^2 - (2x-5y)^2 \\
%		%\text{or, } LHS &= \{(2x+5y) + (2x-5y)\}\{(2x+5y) - (2x-5y)\} \\
%		%\text{or, } LHS &= \{(4x)\}\{(10y)\}\\
%		%\text{or, } LHS &= 20xy
%		\intertext{Compound amount earned by $A$ after $t=2$ years is equal to the Compound amount earned by $B$ after $t=3$ years.}
%		x\left(1+\frac{4}{100}\right)^2 &= (81600-x)\left(1 + \frac{4}{100}\right)^3 \nonumber\\
%		\text{or, } x &= (81600 - x)(1.04) \nonumber\\
%		\text{or, } x(1+1.04) &= 81600(1.04) \nonumber\\
%		\text{or, } x &= \frac{81600*1.04}{2.04} \nonumber\\
%		\text{or, } x &= 41600 \nonumber		
%	\end{align}

%\section{Question 01}
%Prove that if $a, b \text{ and } c$ form a Pythagorean triplet i.e. $a^2 + b^2 = c^2$, then $ka, kb \text{ and } kc$ also form a Pythagorean triplet where $k$ is a real positive constant.
%
%\begin{align}
%	\intertext{Given, } a^2+ b^2 = c^2 \label{eq:given01}\\
%	\intertext{To prove that:}
%	(ka)^2 + (kb)^2 = (kc)^2 \label{eq:tpt01}
%	\intertext{Starting with the LHS of \ref{eq:tpt01}:}
%	LHS = (ka)^2 + (kb)^2 \nonumber\\
%	\text{or, } LHS = k^2 a^2 + k^2 b^2 \nonumber\\
%	\text{or, } LHS = k^2(a^2 + b^2) \nonumber\\
%	\intertext{But from \ref{eq:given01}, we know that} a^2 + b^2 = c^2 \nonumber\\
%	\text{which makes } LHS = k^2c^2 \nonumber
%	\intertext{which is what we set out to prove.}
%	\intertext{Therefore if $a, b, \text{ and } c$ form a Pythagorean triplet then so do $ka, kb \text{ and } kc$. Hence Proved.}  	
%\end{align}
%$\def\num#1{\numx#1}\def\numx#1e#2{{#1}\mathrm{e}{#2}}$ 

\section{Introduction to Scientific Notation for Numerals}

Instead of zeros, for convenience, we like to express numbers in terms of a single digit (which may or may not have fractional or decimal components) multiplied by a power of $10$. Because adults are lazy, instead of writing $10^A$ every time, they write $eA$ to express that the number before it is multiplied by $10^A$. By the way, in your \textit{scientific}$\textsuperscript{\texttrademark}$calculator, you can actually use that, and in fact people use it all the time. Please refer to the examples below.
 
\begin{alignat}{3}
	\intertext{Numbers bigger than $1$ can be written with a positive value of the exponent.}
	700 &= 7*10^2 &=7e2 \nonumber\\
	352 &= 3.52*10^2 &= 3.52e2 \nonumber\\
	5326.6 &= 5.2366*10^3 &= 5.3266e3 \nonumber\\
	32 &= 3.2*10^1 &= 3.2e1 \nonumber\\
	4,900,000,000 &= 4.9*10^9 &= 4.9e9 \nonumber\\
	3.2 &= 3.2*10^0 &= 3.2e0 \label{eq:eq1} 
	\intertext{Although no one writes \Cref{eq:eq1} like that.}
	\intertext{Numbers smaller than $1$ can be written with a negative value of the exponent.}
	0.7 &= 7*10^{-1} &= 7e-1 \nonumber\\
	0.00000256 &= 2.56*10^{-6} &= 2.56e-6 \nonumber\\
	\intertext{Units are no different. Instead of writing $1m = 100cm$, we can write a more efficient $1m = 1e2cm$. Similarly instead of writing $1cm = \frac{1}{100}m$ or $1cm = 0.01m$, we write $1cm = 1e-2m$. This becomes even more important when we are talking about squares or cubes of unit quantities, such as areas and volumes.}
	0.75cm &= 0.75*10^{-2} m &= 0.75e-2 m \nonumber\\
	\text{or, } 0.75cm &= 7.5e-1cm &= 7.5e-3m  \nonumber\\
	(0.75cm)^2 &= (0.75e-2 m)^2 &= 0.5625e-4 m^2 \nonumber\\
	\text{or, } (0.75cm)^2 &= 5.625e-5 m^2 \nonumber
\end{alignat}

\section{Q113 of NCERT Class 8 Mathematics Exemplar}
	Note: Don't be intimidated when I use $\frac{dV}{dt}$ or $\frac{dh}{dt}$ instead of Volume per unit time and height per unit time. It's just some standard notation which you'll probably learn  first about, in class XI. Through this question, I wanted to highlight how using simple scientific notation saves you from the awkwardness of counting zeros or digits after decimal points.
	
	\begin{align}
		\frac{dV}{dt} &= \pi r^2\frac{dh}{dt} \nonumber\\
		\text{Here, } \frac{dV}{dt} &= \frac{22}{7}*(0.75e-2m)^2 * (7m/s) \nonumber\\
		\text{or, } \frac{dV}{dt} &= 22*\Big(\frac{3}{4}e-2m\Big)^2 * (1m/s) \nonumber\\
		\text{or, } \frac{dV}{dt} &= 22*\Big(\frac{9}{16}e-4m^2\Big) * (1m/s) \nonumber\\
		\intertext{In one hour, total volume transported would be:}
		V &= 22*\Big(\frac{9}{16}e-4m^2\Big) * (1m/s) * 3600s \nonumber\\
		\text{or, } V &= 44550e-6m^3 \nonumber
		\intertext{But, $1m^3 = 1000l$. So:}
		V &= 44500e-6m^3 * 1000l/m^3 \nonumber\\
		\text{or, } V &= 44500e-3l \nonumber\\
		\text{or, } V &= 44.5l \nonumber
	\end{align}
	
	Funnily enough NCERT provides the correct value in $l$ but incorrect value in $cm^3$.
\end{document}