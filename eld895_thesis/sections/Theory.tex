\section[Theory]{Theory}
\label{sec:theory}

\textbf{Detrended Fluctuation Analysis}: A method of analyzing real-world time series for self-affinity, i.e. how correlated a signal's future value is to it's past values. Say, $x(t)$ is a natural process, then in order to detrend it, it can first be passed via a low-pass filter to obtain \verb|LPF|$(x(t))$ and the resultant signal be subtracted from the original in order to obtain the detrended version of the signal $\tilde{x}(t)$ or $d(x(t))$:
\begin{equation}
	\tilde{x}(t) \verb| or | d(x(t)) = x(t) - \verb|LPF|(x(t))
\end{equation}

\textbf{Autocorrelation function}: A statistical measure of the correlation of a state variable with a time-lagged version of itself, autocorrelation $c(t, \tau)$ of any detrended physical/natural signal $\tilde{x}(t)$ or $d(x(t))$, should decrease exponentially as the time-lag $\tau$ is increased.
\begin{equation}
	c(x(t), \tau) = \int_{-\infty}^{\infty}x(t)x(t+\tau)dt  
\end{equation}

