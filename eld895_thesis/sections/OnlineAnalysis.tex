\section[Online/Real-time Analysis]{Online Analysis}
\label{sec:online}

On similar lines as \cite{ghanvati01, sanchez01}, we're interested in testing if symptoms of Critical Slowing Down can be detected by a real-time/online analysis of the state variables of a power grid. In other words, we're interested in checking if computing the autocorrelation and variance of real-time PMU data processed over a running window can provide us with Early Warning Signs of an impending instability.

Using the concepts of Bifurcation Theory, a small change in system parameters, such the governor reference power for a generator ($P_{Gen}$) at certain points, can lead to major upsets in the stability of the power grid. We intend to run a simulation in which a system is purposefully stressed (via a near constant linear load increment) as time progresses but many restrictions/safety mechanisms being lifted with the aim of singling-out the cause of bifurcation to a change in $P_{Gen}$(s), in order to best demonstrate that the proposed statistical mechanisms (computing autocorrelations and variances) function well as Early Warning Signs even for slow and steady variations of loads, and not just for sudden changes in state variables caused due to sudden corrective protection mechanisms or the machines not being given `free-range' for chasing load increments due to specified safety limits on maximum allowed generated powers. Below is the set of special conditions for the simulation of the IEEE 9 Bus system:

\begin{enumerate}
	\item The three load points of the system (Buses 5, 6 and 8) were linearly increased in time, at a rate of $\Delta P \%$ per minute plus a small white noise component $\mathcal{N}(0, \sigma_v)$, with every increment happening at $\Delta t$ time intervals. 
	\begin{equation}
		P_{L_i}(t+\Delta t) = P_{L_i}(t)*\left(1+ \frac{\Delta P_{L_i}}{100}\right) + \mathcal{N}(0, \sigma_v)
	\end{equation} 
	Here, we've assigned $\Delta P_{L}$ values randomly between $8-12\%$ for every load bus, $\sigma_v = 0.01$ and $\Delta t = 0.1$ seconds.
	\item Simulation ODE solver solves for the new state variables and parameters for the system every $0.01$ seconds. This means that the simulation output can be likened to a stream of PMU data whose sampling rate is $100$ Hz.
	\item Protection mechanisms were disabled. There will be no remedial/corrective action taken for any drop in bus voltages/grid frequency or any increase in line currents/MVAs.
	\item `Dummy' governors were placed on the three generators (at buses 1, 2 and 3) which can respond instantly to load changes by changing the set reference generation powers $P_{Gen}$(s) with zero time lag.
	\item The generator limits for $P_{Gen_{MAX}}$, $Q_{Gen_{MAX}}$, etc. were removed. Thus the generators had complete freedom to `chase' the load increments at the load buses, including factoring in the extra line-losses.
\end{enumerate} 

It should be noted that while autocorrelation was used in both online/real-time and offline/postmortem analyses, the two usages were different in:

\begin{itemize}
	\item their mode of procuring and processing input data (a running window of an incoming stream of data vs previously stored months/years worth of time series),
	\item the degrees of freedom allowed for its two parameter variables (which out of $t$ and $\tau$ is allowed to be constant),
	\item their theoretically expected output data (autocorrelation is should decrease exponentially with respect to time lag $\tau$ but increase with time $t$ if that the system is being progressively stressed with time)
\end{itemize} 
 