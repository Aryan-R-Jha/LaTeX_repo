\section[Introduction]{Introduction}
\label{sec:intro}

Unlike transient faults in a power grid which can generally be attributed to a sudden but tangible anomaly (corrective outcomes of protection mechanisms, sudden failure of a generator or transformer, line faults), steady state instabilities may remain undetected until they accumulate over time to manifest as a major upset to the grid \cite{entsoeReportGridCollapseContinentalEurope2021Jan} and/or make the grid less robust/more susceptible to collapses \cite{schafer01}. The causes of these disturbances is often an accruing of stochastic perturbations in the state variables of the grids when it is already stressed by an increased power demand \cite{rosehartBifurcationAnalysisOfVariousPowerSystemModels}.  The accrual of stochastic perturbations in turn can be caused due to various physical phenomena, including measurement noises, distributed renewable generation fluctuations \cite{adeen01}, sudden gaps in power demand and supply due to power trading \cite{schafer01}, strict operational frequency deadbands \cite{vorobev01, francesca01}, etc.
 
While modeling every significant possible source of stochastic disturbance can be difficult or perhaps even outright impossible, at least their detection can be made through model free data-driven statistical analysis, enabling early detection of grid stability problems for a timely course-corrective action \cite{schafer01, sanchez01, ghanvati01}.

Bifurcation Theory \cite{nathanKutzNotesOnBifurcationTheoryAndNormalForms, rosehartBifurcationAnalysisOfVariousPowerSystemModels, chenBifurcationsAndChaosInEngineering, mohlerDyanmicsAndControlPartOne} helps explain the erratic functioning of stressed dynamical systems such as the power grid, and the theory of Critical Slowing Down \cite{schefferEarlyWarningSignalsForCriticalTransitions} lists tangible quantitative analysis tools which can help us detect an impending `bifurcation' (blackout) in the power grid.

In this thesis, we first investigate various real-world grid frequency time series archives on their robustness against minor disturbances and any kind of long-standing stability problems in them, through the use of bulk distribution probability density functions and autocorrelation decay plots. We refer to this analysis as Offline/Postmortem analysis as the input data used is sampled for a long period (several months or years). Next, we investigate the effectiveness of two statistical parameters computed in real-time listed out as per the theory of Critical Slowing Down, namely Autocorrelation and Variance as Early Warning Sign Indicators of an approaching bifurcation in a power grid. We label this analysis as the Online/Real-time analysis as the input data here is only in instantaneously available from a stream of PMU data.
