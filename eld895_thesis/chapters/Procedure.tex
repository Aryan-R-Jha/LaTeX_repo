\section[Procedure]{Procedure}
\label{sec:procedure}

\begin{frame}{Procedure: Simulation}
	\begin{itemize}
		\item Created and simulated the IEEE 9 Bus System in PSSE 34.
		\item Added stochastic disturbance to the loads (at Bus 5, 6 and 8) via white noise modeled as
			\begin{equation}
				(P_L)_{i}[k] = (P_L)_{i}[k-1] + N(0, \sigma^2)
			\end{equation}
		for load bus $i$ at discrete sample $k$.
		$\sigma = 0.01$ pu. 	
		\item In order to drive the power grid towards bifurcation, steadily increased the three loads at different rates, between 20\% to 30\% per minute.
		\item Ran a time analysis simulation until critical bifurcation attained. 
		\item Extracted the bus voltages. 
	\end{itemize}

\end{frame}

\begin{frame}{Procedure: Detrending Bus Voltages}
	Passed the voltage signals through a Low Pass Filter in order to capture the slow changing trends not an effect of CSD. 
	Eg. Change in bus voltages due to the gradual increasing of loads.
	Gaussian Kernel Smoothing Filter was used for the same.
	\begin{equation}
	\label{eq:gks}
		h(n, \sigma_f) = \frac{1}{\sqrt{2 \pi} \sigma_f} \exp^\frac{-n^2}{2 \sigma_f^2}
	\end{equation}
	where $\sigma_f$ can be varied between $5$ to $10$ .
	\\ The detrended signal is obtained by subtracting the filtered signal from the original signal.	
	\begin{equation}
	\label{eq:detrending}
		d[x] = x[k] - GKS(x[k])
	\end{equation} 
\end{frame}

\begin{frame}{Procedure: Applying AR1 and Variance}
	
	Windows varying between $W = 1$ second to $W = 40$ seconds were used for computing autocorrelations and variances.
	
	For every window, Autoregressive Model of Order 1 was fitted onto the detrended voltages using least squares of error approach.
	\begin{equation}
	\label{eq:autocorrDef}
	d[k] = a_1 d[k-1] + e[k]
	\end{equation} 
	
	Variances were also computed for every window.
	\begin{equation}
	\label{eq:varDef}
	\sigma^2 = \frac{1}{n_k} \sum_{k=1}^{n_k} d[k]^2
	\end{equation} 
\end{frame}