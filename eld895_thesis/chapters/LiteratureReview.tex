\section[Literature Review and Objectives]{Literature Review and Objectives}
\label{sec:litt}

For offline analysis of the grids, archived frequency time-series data of several real-world grids was downloaded from these websites and/or papers: \cite{lrydin01, lrydinGithub, tokyo2017, tokyo2020, nordic2018, nordic2019, ce2019, ce2020,  ukNationalGridESOData}. Schafer et al's paper \cite{schafer01} was referred to for the analysis of these grids. Their paper  analyzes how almost all grids show a significant level of deviation from the commonly used assumption that power demand fluctuations follow the Ornstein-Uhlenbeck's Process in which the state variable follows the Gaussian distribution around a nominal mean and a bounded standard deviation. They also explain some of the causes of detected instabilities for various power grids, including measurement noise and energy trading fluctuations.

Most of the recent literature analyzing the effects of Critical Slowing Down on various dynamical systems including but not limited to: power grids, ecological population dynamics, predator-prey ecosystems, prediction of epileptic seizures in patients, climate systems, financial markets, prediction of conversion of vegetation area into deserts, and so on, credit the review made by Scheffer et al in \cite{schefferEarlyWarningSignalsForCriticalTransitions}. The paper lists out systems in which Critical Slowing Down has been observed and provides accessible mathematical explanations behind its working, such as why is there an increase of autocorrelation and variance of state variables of real-world physical processes as the system approaches a `critical bifurcation'. The mathematical term has also been called appropriately as `critical transition' or `tipping point' by the author of \cite{kuehnMathematicalFrameworkForCriticalTransitions}, whose paper explains various normal forms of bifurcations (Fold, Hopf, Saddle Node, Transcritical, Pitchfork) via the concept of fast-slow stochastic systems. For developing a working understanding of bifurcation theory, university lecture notes by \cite{nathanKutzNotesOnBifurcationTheoryAndNormalForms} and books \cite{chenBifurcationsAndChaosInEngineering, mohlerDyanmicsAndControlPartOne} were utilized.  

For real-time/online analysis, authors in \cite{ghanvati01} have utilized PSAT \cite{psatMilano} to simulate a steadily stressed power grid and have demonstrated that the computation of autocorrelation of detrended bus voltages and the computation  of variance of detrended line currents can function as reliable Early Warning Signs of increasing instability. The detrending is required in order to filter any measurement noise from the data, which may skew the computed statistical parameters towards bogus values. Adeen et al's paper \cite{adeen01} simulated several Stochastic Differential Equations based on the Ornstein-Uhlenbeck's Process with different values of $\alpha$ (autocorrelation coefficient) and analyzed their Fourier Spectrums to conclude that an increased autocorrelation does in fact lead to a greater amplitude of noise and therefore a higher risk of instability in a power system. Authors in \cite{sanchez01} tested various power grids which were driven towards bifurcation and demonstrated that an increase of autocorrelation and variance values of bus voltages (tested in simulation) and grid frequency (tested on the time-series data measured at the Bonneville Power Administration minutes before the blackout of 10 August 1996) can reliably predict the impending bifurcation early enough for mitigating actions to be taken by the grid operator.

For simulation implementations in PSSE 34.3, the community run website run by Jervis Whitley \cite{psspyWebsite} was very helpful.

This thesis aims to highlight how statistical analysis can help predict/observe/detect indicators to instabilities in power grids through both offline and online studies while making the least number of assumptions about the grid models themselves due to its data-driven approach. Statistical analysis can detect both lingering instability causing agents in the grids through Offline/Postmortem Analysis as well as predict any impending blackout/`bifurcation' in the grid through Online/Real-time Analysis. Typical power grid state variables such as Bus Voltages, Line Currents/MVAs and Grid Frequencies obtainable from a stream of PMU data may be used as inputs for such data-driven analysis. Tools used for Offline/Postmortem Analysis are visual inspection of bulk-distribution PDFs and estimating grid damping constants from autocorrelation decay curves. Tools used for Online/Real-time Analysis involve computing fixed-lag autocorrelation and variance of the filtered detrended fluctuations.

All simulations were done in Siemens PSSE 34.3 in conjunction with Python 2.7 (for writing automation scripts). All data analysis was conducted in MATLAB 2022a. A working implementation for anyone interested may be downloaded via [\href{https://t.ly/HwAT}{Simulation}, \href{https://t.ly/e1f9}{Offline Analysis}, \href{https://t.ly/u_Mp}{Online Analysis}]