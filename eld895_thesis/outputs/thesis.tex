\documentclass[Other]{iitddiss}

\newcommand{\degreeName}{Masters of Science in Electrical Engineering}
\program{\degreeName}

 \usepackage{t1enc}

\usepackage{graphicx}
\usepackage{hyperref} % hyperlinks for references.
\usepackage{amsmath} % easier math formulae, align, subequations \ldots
\usepackage[english]{babel}
\usepackage[utf8]{inputenc}
\usepackage{natbib}
\usepackage{fancyhdr}
 \linespread{1.2}

\pagestyle{fancy}
\renewcommand{\sectionmark}[1]{\markright{\thesection\ #1}}

\fancyhf{}

\rhead{\fancyplain{}{\thepage}} % predefined ()
\lhead{\fancyplain{}{\rightmark}} % 1. sectionname, 1.1 subsection name etc
\newcommand{\orgName}{Indian Institute of Technology Delhi}
\cfoot{\textcopyright \text{ } \the\year, \emph{\orgName}}
\renewcommand{\footrulewidth}{0.4pt}
\begin{document}


%%%%%%%%%%%%%%%%%%%%%%%%%%%%%%%%%%%%%%%%%%%%%%%%%%%%%%%%%%%%%%%%%%%%%%
% Title page
\newcommand{\titleName}[0]{Data Analysis for Predicting Instabilities in Power Systems}
%\title{Data Analysis for Predicting Instabilities in Power Systems}
\title{\titleName}

\newcommand{\authorName}{Aryan Ritwajeet Jha}
\author{\authorName}

\newcommand{\advisorName}{Prof. Nilanjan Senroy}
\advisor{\advisorName}

\newcommand{\advisorPosition}{Professor}

\newcommand{\entryNumber}{2020EEY7525}
\entrynumber{\entryNumber}

\newcommand{\dateFixed}{July 2022}
\date{\dateFixed}

\newcommand{\departmentName}{Department of Electrical Engineering}
\department{\departmentName}

\newcommand{\placeName}{Hauz Khas, New Delhi, 110016, India}
%\nocite{*}
\maketitle

%%%%%%%%%%%%%%%%%%%%%%%%%%%%%%%%%%%%%%%%%%%%%%%%%%%%%%%%%%%%%%%%%%%%%%
% Certificate
\certificate

\vspace*{0.5in}

\noindent This is to certify that the thesis titled {\bf \titleName{}}, submitted by {\bf \authorName{}},
  to the {\bf \orgName{}} for
the award of the degree of {\bf \degreeName{}}, is a bona fide
record of the research work done by him under our supervision. The
contents of this thesis, in full or in parts, have not been submitted
to any other Institute or University for the award of any degree or
diploma.

\vspace*{1.5in}

%\begin{singlespacing}
\hspace*{-0.25in}
\parbox{3.0in}{
\noindent \advisorName{} \\
\noindent \advisorPosition{} \\
\noindent \departmentName{}\\
\noindent \orgName{} \\
}
\hspace*{0.0in}
%\end{singlespacing}
\vspace*{0.40in}
\noindent \placeName{}\\
Date: \dateFixed{}


%%%%%%%%%%%%%%%%%%%%%%%%%%%%%%%%%%%%%%%%%%%%%%%%%%%%%%%%%%%%%%%%%%%%%%
% Acknowledgements
\acknowledgements



%%%%%%%%%%%%%%%%%%%%%%%%%%%%%%%%%%%%%%%%%%%%%%%%%%%%%%%%%%%%%%%%%%%%%%
% Abstract

\abstract

\noindent KEYWORDS: \hspace*{0.5em} \parbox[t]{4.4in}{}

\vspace*{24pt}

\pagebreak

%%%%%%%%%%%%%%%%%%%%%%%%%%%%%%%%%%%%%%%%%%%%%%%%%%%%%%%%%%%%%%%%%
% Table of contents etc.

\begin{singlespace}
\tableofcontents
\thispagestyle{empty}

\listoftables
\addcontentsline{toc}{chapter}{LIST OF TABLES}
\listoffigures
\addcontentsline{toc}{chapter}{LIST OF FIGURES}
\end{singlespace}


%%%%%%%%%%%%%%%%%%%%%%%%%%%%%%%%%%%%%%%%%%%%%%%%%%%%%%%%%%%%%%%%%%%%%%
% Abbreviations
\abbreviations

\noindent
\begin{tabbing}
xxxxxxxxxxx \= xxxxxxxxxxxxxxxxxxxxxxxxxxxxxxxxxxxxxxxxxxxxxxxx \kill
\textbf{IITD}   \> Indian Institute of Technology, Delhi \\
\textbf{RTFM} \> Read the Fine Manual \\
\end{tabbing}

\pagebreak

\pagebreak
\clearpage

% The main text will follow from this point so set the page numbering
% to arabic from here on.
\pagenumbering{arabic}


%%%%%%%%%%%%%%%%%%%%%%%%%%%%%%%%%%%%%%%%%%%%%%%%%%
% Introduction.

\chapter{INTRODUCTION}
\label{chap:intro}

\section{Example Figures and tables}

Fig.~\ref{fig:iitd} shows a simple figure for illustration along with
a long caption.  The formatting of the caption text is automatically
single spaced and indented.  Table~\ref{tab:sample} shows a sample
table with the caption placed correctly.  The caption for this should
always be placed before the table as shown in the example.


\begin{figure}[htpb]
  \begin{center}
    \resizebox{50mm}{!} {\includegraphics *{iitd_logo.png}}
    \resizebox{50mm}{!} {\includegraphics *{iitd_logo.png}}
    \caption {Two IITD logos in a row.  This is also an
      illustration of a very long figure caption that wraps around two
      two lines.  Notice that the caption is single-spaced.}
  \label{fig:iitd}
  \end{center}
\end{figure}

\begin{table}[htbp]
  \caption{A sample table with a table caption placed
    appropriately. This caption is also very long and is
    single-spaced.  Also notice how the text is aligned.}
  \begin{center}
  \begin{tabular}[c]{|c|r|} \hline
    $x$ & $x^2$ \\ \hline
    1  &  1   \\
    2  &  4  \\
    3  &  9  \\
    4  &  16  \\
    5  &  25  \\
    6  &  36  \\
    7  &  49  \\
    8  &  64  \\ \hline
  \end{tabular}
  \label{tab:sample}
  \end{center}
\end{table}

\end{document}
