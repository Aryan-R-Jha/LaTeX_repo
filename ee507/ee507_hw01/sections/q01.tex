\section{Problem 1}
 
 All of you travelled to Pullman and WSU to begin your education here (maybe by airplane, maybe by car, maybe even by bus or on foot if you grew up in the area). Please consider your latitude (how far North or South you are) as a function of time on this journey. Before you started travelling, this time signal was subject to uncertainty. Please sketch two possible instances of this signal, explaining why each result may occur.
 
\subsection{Solution}

India and the West Coast of United States are almost at the opposite ends of the globe. Geographically, this leads to interesting possibilities as flights can be optimally made from both sides across the globe.

Flights from New Delhi, India to Pullman, WA, US can be broadly classified into two categories based on which ocean they fly over during their trip. Flights can come via the Atlantic Ocean or via the Pacific Ocean.

\cref{tab:AtlanticRoute} and \cref{tab:pacificRoute} show the latitude coordinates of the plane with respect to time and location, including layover times.

\cref{fig:signalMaps} shows the two signal curves of latitudes vs time.

\begin{table}[h]
	\centering
	\caption{Atlantic Ocean Route}
	\label{tab:AtlanticRoute}
	\begin{tabular}{llll}
		\toprule
		City (Country) & Latitude & Journey Time (hrs) & Layover Time (hrs)\\
		\midrule
		New Delhi (India)& 28.6139° N & 0 & 0\\
		Doha (Qatar) & 25.2854° N & 4 & 3\\
		Settle, WA (US) & 47.6062° N & 14 & 3 \\
		Pullman, WA (US) & 46.7298° N & 1 & 0\\
		\bottomrule
	\end{tabular}
\end{table}

\begin{table}[h]
	\centering
	\caption{Pacific Ocean Route}
	\label{tab:pacificRoute}
	\begin{tabular}{llll}
		\toprule
		City (Country) & Latitude & Journey Time (hrs) & Layover Time (hrs)\\
		\midrule
		New Delhi (India)& 28.6139° N & 0 & 0 \\
		Seoul (South Korea) & 37.5665° N & 7 & 14 \\
		Settle, WA (US) & 47.6062° N & 10 & 3 \\
		Pullman, WA (US) & 46.7298° N & 1 & 0 \\
		\bottomrule
	\end{tabular}
\end{table}

\begin{figure}[H]
	\centering
	\import{../figures/}{signalMapPdf.pdf_tex}
	\caption{Latitude vs Time Plots for the two routes.}
	\label{fig:signalMaps}
\end{figure}

\noindent\rule{\textwidth}{1pt}
\newpage