\appendix
\section{Brownian Motion and the Ornstein-Uhlenbeck Process}
Scientist Robert Brown quoted the term `Brownian motion' to describe the random yet not chaotic motion of pollen grains suspended in water \cite{brownianMotion}.

The Ornstein-Uhlenbeck Process (named in honour of Physicists Leonard Salomon Ornstein and George Eugene Uhlenbeck) is a special case of the Brownian Motion.  

\section{More Examples for Critical Bifurcation}

\begin{figure}[!ht]
	\centering
		\centering
		\import{../figures/}{bifurcationTranscriticalPdf.pdf_tex}\quad
		\import{../figures/}{bifurcationTranscriticalNegativePdf.pdf_tex}
	\caption{Bifurcation diagrams for the normal forms of the Transcritical Bifurcation: \\$\frac{dx}{dt} = \mu x - x^2$ and $\frac{dx}{dt} = \mu x + x^2$ \\For $\mu \leq 0$, there are two fixed points (equilibrium points), $x=0$ and $x = \mu$ (or $x = -\mu$), out of which one is stable and the other unstable. Upon reaching the critical `tipping' point $\mu=0$, the two fixed points or equilibrium solutions exchange their stabilities. The tipping point $\mu=0$ is a bifurcation point.}
	\label{fig:bifTranscritical}
\end{figure}