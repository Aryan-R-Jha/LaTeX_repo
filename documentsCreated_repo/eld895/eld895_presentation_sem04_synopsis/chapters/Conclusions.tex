\section[Conclusions]{Summary and Conclusions}
\label{sec:concl}

\begin{frame}[allowframebreaks]{Conclusions}
	\noindent\textbf{Offline/Postmortem Analysis}
	\begin{itemize}
		\item Frequency time-series for months/years of data obtained from various real-world grids were converted into probability distribution function plots and autocorrelation decay plots ($c(\tau)$ vs $\tau$ plots).
		\item Visual inspection of the probability distribution function plots provided many insights into the presence of long-standing steady-state instabilities in the grid as well as the grid's resilience against any additional instability causing agents. Generally the PDFs of the more robust grids such as the RTE (France) and Continental European grids were mostly Gaussian except that they had heavier tails, whereas the smaller or island grids, such as the Mallorcan (Spain) grid had multiple peaks, skewed distributions and thus an overall visible deviation from Gaussianity which explains their higher susceptibility to steady-state deviations and thus a greater degree of vulnerability to grid failures. 
		\item For most grids, the autocorrelation functions exponentially decayed with respect to time lag $\tau$ for smaller values of $\tau$ but certain grids showed significant deviation from the expected norm. For example the Continental European and UK grids showed a spike in autocorrelation decay function at time lags of every 15 minutes. This spike, which indicates an inherent instability causing agent in the grid systems, can be attributed to their 15 minute power trading intervals. Unlike the amount of transacted power which is suddenly varied every 15 minutes, the power grids, being dynamical systems cannot instantly adjust to the new power settings and thus the sudden imbalance of supply and demand leads to transients in the grid state variables.
		\item Autocorrelation decay curves of other grids (Nordic, Japan, US-Western Interconnection) initially decreased exponentially but later followed between a very slowly decaying or almost constant curve with respect to $\tau$. This can be attributed to measurement noise in the frequency detection.
		\item From the initial exponential decay of the curves, semi-log graphs were plotted and their inverse correlation times ${t_{corr}}^{-1}$ were obtained. As per the Ornstein-Uhlenbeck Process this inverse correlation time can be likened to the damping constant $\alpha$ of the grids. As per our theoretical expectations, the bigger and more robust grids had higher values of $\alpha$ compared to the smaller, islanded grids.
	\end{itemize}
\end{frame}

\begin{frame}[allowframebreaks]{Conclusions}
	\noindent\textbf{Online/Real-time Analysis}
	\begin{itemize}
		\item The IEEE 9 Bus System was progressively stressed in a time-domain simulation until `bifurcation' was achieved \cite{sanchez01}. In terms of implementation, `bifurcation' was concluded to have taken place when the simulation solver could no longer converge to a solution without violating convergence thresholds. PSSE 34.3 simply calls out this occurrence as `Network Not Converged'.
		\item The bus voltages were detrended with the help of a low pass filter, and their variance $\sigma^2$ as well as autocorrelations $c(t, \tau)$ with a fixed time lag $\tau = 1$ instance were computed over a running window. 
		\item A new statistical parameter, called the Modified Kendall's $\tau$ Correlation Coefficient (MKTCC) was employed to check if the increase in the autocorrelations and variances was statistically significant. The reason for using a modified version of the normally used Kendall's $\tau$ Correlation Coefficient was to accommodate for the degree of certainty/confidence in predicting the correlation apart from the absolute value of correlation itself.
		\item Both autocorrelation and variance were found to be appropriate Early Warning Sign Indicators for an impending bifurcation, predicting the event minutes earlier. 
	\end{itemize}
\end{frame}




