\section[Motivation and Objectives]{Motivation and Objectives}
\label{sec:obj}

\begin{frame}{Objectives}
	This thesis aims to highlight how statistical analysis can help predict/observe/detect steady state instabilities in power grids through both offline and online studies while making the least number of assumptions about the grid models themselves due to its data-driven approach. Statistical analysis can detect both lingering instability causing agents in the grids through Offline/Postmortem Analysis as well as predict any impending blackout/`bifurcation' in the grid through Online/Real-time Analysis.
\end{frame}

\begin{frame}{Objectives}
	 Typical power grid state variables such as Bus Voltages, Line Currents/MVAs and Grid Frequencies obtainable from a stream of PMU data may be used as inputs for such data-driven analysis. Tools used for Offline/Postmortem Analysis are visual inspection of bulk-distribution PDFs and estimating grid damping constants from autocorrelation decay curves. Tools used for Online/Real-time Analysis involve computing fixed-lag autocorrelation and variance of the filtered detrended fluctuations.
\end{frame}

\begin{frame}{Objectives}
	All simulations were done in Siemens PSSE 34.3 in conjunction with Python 2.7 (for writing automation scripts). All data analysis was conducted in MATLAB 2022a. A working implementation for anyone interested may be downloaded via [\href{https://t.ly/HwAT}{Simulation}, \href{https://t.ly/e1f9}{Offline Analysis}, \href{https://t.ly/u_Mp}{Online Analysis}]
\end{frame}

