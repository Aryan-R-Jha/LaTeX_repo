\section{Problem 2}

\begin{enumerate}[2a.]
	\item If two events $A$ and $B$ are independent, is it necessarily true that $A$ and $\bar{B}$ are independent Please prove or give a counterexample.
	\item If events $A$ and $B$ are independent, and events $A$ and $C$ are independent, is it always true that events $B$ and $C$ are independent? Please prove or give a counterexample.
	\item If two events are disjoint, are they always, sometimes, or never independent? Please explain.
\end{enumerate}

\subsection{Solution}
\begin{enumerate}[2a.]
	\item
	\text{From Law of Total Probability, we know that:}
	\begin{align}
		\text{Given that, } P(AB) &= P(A)P(B) \label{eq:given1} \\
		\text{To check if: } P(A\bar{B}) &= P(A)P(\bar{B}) \label{eq:toCheck1} \\
		A &= AB + A\bar{B} \label{eq:LOTPAB}\\
		\text{or, } P(A) &= P(AB) + P(A\bar{B}) \nonumber\\
		\text{or, } P(A) &= P(A)P(B) + P(A\bar{B}) \nonumber\\
		\text{or, } P(A) - P(A)P(B) &= P(A\bar{B}) \nonumber\\
		\text{or, } P(A){1-P(B)} &= P(A\bar{B}) \nonumber\\
		\text{or, } P(A)P(\bar{B}) &= P(A\bar{B}) \nonumber
	\end{align}
	\text{which is exactly what we set to prove i.e. \cref{eq:toCheck1}.} \\
	\text{$\implies$ If $A$ and $B$ are independent, then so are $A$ and $\bar{B}$.}\\
	Hence Proved.\Laughey
	
	\item No. Assume the trivial cases $B = C$ or $B = \bar{C}$. Even when $P(A|B) = P(A)$ and $P(A|C) = P(C)$ as given, $P(B|C) = 1 \neq P(B)$ in the first case and $P(B|C) = 0 \neq P(B)$ in the second.
	
	\item Disregarding the trivial case when at least one of the events is impossible, two disjoint events can NEVER be independent. Conceptually this makes sense, as say, if events $A$ and $B$ with non-zero probabilities are completely disjoint, then we could say that $B \in \bar{A}$, thus $P(A|B) = 0 \neq P(A)$ and vice versa $P(B|A) = 0 \neq P(B)$, which basically violates the condition of independence.
\end{enumerate} 
\noindent\rule{\textwidth}{1pt}