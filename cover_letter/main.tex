% Title:  An unofficial letter class for Washington State University
% Author: Adam Erickson, Postdoctoral Researcher, Strigul Laboratory
% Notes:  Use the `bw` option for improved black-and-white printing.
%         One may change the defaults in the `coverletter.cls` file
%         or simply update the from* fields below.
\documentclass[color, 10pt, letterpaper]{coverletter}

\usepackage{lipsum}
\usepackage{enumitem}
\usepackage{tikzsymbols}
\usepackage{amssymb}
\usepackage{hyperref}

\hypersetup{
    colorlinks=true,
    linkcolor=blue,
    filecolor=magenta,      
    urlcolor=cyan,
    pdftitle={Overleaf Example},
    pdfpagemode=FullScreen,
    }
    
\update{%
    toname={Nitivia Jones},
    totitle={},
    torole={System DSO  \\Senior International Student Advisor},
    todept={International Programs (IP)\\International Student and Scholar Services},
    touni={Washington State University, Pullman},
    tocity={Pullman, WA, 99163},
    tocountry={United States}
}

\begin{document}

    Through this letter, I wish to express my interest in applying to be a mentor in the International Student Peer Mentor Program at Washington State University (WSU), Pullman.

    This program is something which I've personally benefited from, as a mentee, when I arrived here last Fall. I would be honoured if I'm given the opportunity to contribute back to this program.

    \textbf{Why am I a good fit to be a mentor?}
    I like to think of myself as a very friendly and gregarious person who loves interacting with people of different cultures, nationalities, age-groups and academic backgrounds. I'm the guy who knows a guy. As a representative of WSU for new students, my utmost commitment is to create a welcoming and reassuring environment. I am driven to showcase the inherent friendliness of Pullman and WSU, ensuring that newcomers feel at ease from the moment they arrive. Additionally, I aim to provide comprehensive information about the abundant facilities and infrastructure that are readily accessible to support their academic and personal journey. Students can look forward to me as amongst the first of their friends here.

    \textbf{What kind of mentees should especially look forward to have me as their mentor?}
    \begin{itemize}[noitemsep,topsep=0pt]
        \item \textbf{Career-oriented, academically serious students} I'm a bit old school in my approach, and I think that a university student's top priority should be gain the necessary skills required to excel in the career of their choice. I help fellow students with advice on their courses, projects, internships, industry jobs, higher studies and everything in between.
        \item \textbf{People-loving, diversity embracing students} A huge upside of studying in a university in the United States is the sheer number of different people you get to meet, both from within the country as well as the whole world. I feel privileged to have made friends with students from so many countries, and having learned from their cultures. I would encourage incoming students to take full advantage of this diversity. Through my company, students will be able to broaden their social networks significantly, fostering connections with individuals from diverse backgrounds and experiences.
        \item \textbf{Foodies} I'm a pretty competent cook. I've hosted dozens of people at my apartment for some great meals. While I generally enjoy making Indian/South-Asian cuisine, for people who can't handle the spices, I make equally good Italian-American food.
    \end{itemize}

     I should clarify that these traits are NOT necessary prerequisites for a potential mentee to choose me as their mentor, although I will try to nudge them towards these. \Laughey

    I thank you for your consideration of my application. While my attached resume will have all facts associated with me, it's unfortunately not really tailored to sell me as anything apart from an Electrical Engineering student. So, if you have any further questions for me, please don't hold back, reach out to me via my \href{mailto:aryan.jha@wsu.edu}{email} or \href{tel:509-338-8770}{phone}.
\end{document}
