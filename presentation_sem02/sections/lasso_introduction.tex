\section[Introduction]{Introduction}
\label{sec:lasso_introduction}

\begin{frame}[fragile]{Forced Oscillations in Power Systems}
	\begin{tabularx}{\textwidth}{
			@{\hspace{1.5em}}% Space for left bullet
			>{\leavevmode\raggedright}% Left bullet + formatting of column
			X% Left column specification
			@{\quad\hspace{1.5em}}% Space between columns + right bullet space
			>{\leavevmode\raggedright\arraybackslash}% Right bullet + formatting of column
			X% Right column specification
			@{}% No column space on right
		}
		\textbf{Natural Oscillations} & \textbf{Forced Oscillations}\\
		\toprule
		A sudden, out-of-trend, high magnitude change in a state variable(s) causes blackouts. & 
		Accumulation of several seemingly minor trends in state variables over time, ultimately leading to a \textcolor{red}{critical point} where a small change could cause blackouts.\\
		Chief parameters of concern are ROCOF, frequency nadir, steady-state frequency deviation. & 
		\textcolor{red}{Autocorrelation} and covariance are some of the commonly used parameters for prognosis.\\
		Inertia is a fundamental parameter here. & Inertia plays a minor role here.\\
		\bottomrule
	\end{tabularx}
\end{frame}

\begin{frame}{Bifurcations and Critical Slowing Down}
	\textbf{Bifurcation}: A qualitative change in the `motion' of a dynamical System due to a quantitative change in one of its parameters. Serious bifurcations, called \textcolor{red}{Critical Bifurcations}, cause the system to become unstable from stable.
\end{frame}

\begin{frame}{Bifurcations and Critical Slowing Down}
	\textbf{Critical Slowing Down}: Dynamical Systems exhibit early statistical warning signs before collapsing:
	
	\begin{itemize}
		\item Increased recovery times from perturbations.
		\item Increased signal variance from the mean trajectory.
		\item Increased flicker and asymmetry in the signal
	\end{itemize}
	
	The above three properties can be identified by increasing variance and autocorrelation in time-series measurements taken from the system.
\end{frame}
