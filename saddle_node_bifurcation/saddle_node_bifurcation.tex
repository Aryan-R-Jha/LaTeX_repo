\documentclass{article}



\begin{document}

	
	\title{Saddle-node Bifurcation: The First Steps in Checking the Stability of Equilibrium Points}
	\author{Aryan Ritwajeet Jha}
	\maketitle
	
	Below is the equation for the Saddle-node Bifurcation: 
	\begin{equation}
		\label{eq:saddleNode}
		\frac{dx}{dt} = \mu - x^2
	\end{equation}
	
	which has equilibrium points at
	
	\begin{equation*}
		$x_e = x_{e_1} = +\sqrt\mu$ \\
		and \\
		$x_e = x_{e_2} = -\sqrt\mu$
	\end{equation*} 
	 such that:
	
	\begin{equation}
		\label{eq:saddleNodeEqui}
		\frac{dx_e}{dt} = \mu - x_e^2 = 0 
	\end{equation}
	
	
	Perturbing $x$ around an equilibrium point $x_e$ by a small value $\tilde x$, we can rewrite equation (\ref{eq:saddleNode}) as:
	
	\begin{equation}
		\label{eq:saddleNodePerturbed}
		\frac{d(x)}{dt} = \frac{d(x_e + \tilde x)}{dt} =\frac{dx_e}{dt} + \frac{d \tilde x}{dt} = \frac{d \tilde x}{dt} 
	\end{equation}
	and putting $x = x_e + \tilde x$ in the RHS of equation (\ref{eq:saddleNode}), we get:
	
	\begin{equation}
		\label{eq:saddleNodePerturbed2}
		\frac{d\tilde x}{dt} = \mu - (x_e+\tilde x)^2 \\
		\frac{d\tilde x}{dt} = \mu - x_e^2 - -2x_e\tilde x - \tilde x^2
	\end{equation}
	
	Using equation (\ref{eq:saddleNode}) and equation (\ref{eq:saddleNodeEqui}) in equation (\ref{eq:saddleNodePerturbed2}) and neglecting the small last term in the RHS, we get:
	
	\begin{equation}
		\frac{d\tilde x}{dt} = -2x_e\tilde x
	\end{equation}

\end{document}

