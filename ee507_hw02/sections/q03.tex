\section{Problem 3}

A randomly selected student in an engineering class id diligent with probability $0.3$, and lazy with probability $0.7$. A diligent student completes all of her homework with probability $0.9$ and does not do so with probability $0.1$. Meanwhile, a lazy student completes all homework with probability $0.5$ and does not do so with probability $0.5$. A student who completes all homework receives an $A$ grade with probability $0.6$, and receives a $B$ grade with probability $0.4$ (irrespective of whether the student was diligent or lazy). A student who doesn't complete all homework receives an $A$ grade with probability $0.3$, a $B$ grade with probability $0.4$ and an $F$ grade with probability $0.3$ (again, irrespective of whether the student was diligent or lazy).\\
Please answer the following questions.\\

\begin{enumerate}[3a.]
	\item What is the probability that a randomly selected student is diligent, completes all their homework and receives an $A$ grade?
	\item What is the probability that the student completes all of their homework?
	\item What is the probability that the student receives an $A$ grade?
	\item Given that a randomly selected student completed all their homework, what is the probability that the student was lazy?
	\item Given that a randomly-selected student received an $A$ grade, what is the probability that the student is lazy?
	\item Is the event that a student passes the course (receives an $A$ or $B$ grade) independent of the event that the student is lazy?
	\item Is the event that a student receives a $B$ grade independent of the event that the student is lazy? 
\end{enumerate}

\subsection{Solution}

Experiment S: A student is randomly selected from an engineering class.

\begin{align}
	\text{Let, Event } D &= \set{\text{"Selected student is Diligent"}} \nonumber\\
	\text{$\implies$ Event } \bar{D} &= \set{\text{"Selected student is Lazy"}} \nonumber\\
	\text{Let, Event } H &= \set{\text{"Selected student did all of their homework"}} \nonumber\\
	\text{$\implies$ Event } \bar{H} &= \set{\text{"Selected student did NOT do all of their homework"}} \nonumber\\
	\text{Let, Event } A &= \set{\text{"Selected student receives an $A$ grade."}} \nonumber\\
	\text{Let, Event } B &= \set{\text{"Selected student receives a $B$ grade."}} \nonumber\\
		\text{Let, Event } F &= \set{\text{"Selected student receives an $F$ grade."}} \nonumber \Sadey
\end{align}

For reference, \cref{fig:decisionTree} is a graph which displays all possible variations of a randomly chosen student (or in terms of probability theory all events with non-zero probabilities).

\begin{figure}[H]
	\centering
	\begin{adjustbox}{max width=\textwidth}
		\import{../figures/}{decisionTree.pdf_tex}
	\end{adjustbox}
	\caption{Decision Tree representing all possible types of students selected.}
	\label{fig:decisionTree}
\end{figure}

\begin{enumerate}[3a.]
	\item
		\begin{align}
			\itt{Using conditional probability theorem:}
			P(DHA) &= P(A|HD)P(H|D)P(D) \nonumber\\
			\text{or, } P(DHA) &= (0.6)(0.9)(0.3) \nonumber\\
			\text{or, } P(DHA) &= 0.162 \label{eq:P_DHA}
		\end{align}
	\item
		\begin{align}
			\itt{Using Law of Total Probability:}
			P(H) &= P(H|D)P(D) \nonumber\\
			\text{or, } P(H) &= P(H|D)P(D) + P(H|\bar{D})P(\bar{D}) \nonumber\\
			\text{or, } P(H) &= (0.9)(0.3) + (0.5)(0.7) \nonumber\\
			\text{or, } P(H) &= 0.62 \label{eq:P_H}
		\end{align}
	\item
		\begin{align}
			\itt{Using Law of Total Probability:}
			P(A) &= P(A|\bar{D}\bar{H})P(\bar{D}\bar{H}) + P(A|\bar{D}H)P(\bar{D}H) \nonumber\\
			{} &\quad+ P(A|D\bar{H})P(D\bar{H}) + P(A|DH)P(DH) \nonumber\\
			\itt{Using  $P(DH) = P(H|D)P(D)$ and so on for other terms:}
			\orr P(A) &= (0.3)(0.7)(0.5) + (0.6)(0.7)(0.5) \nonumber\\
			{} &\quad+ (0.3)(0.3)(0.1) + (0.6)(0.3)(0.9) \nonumber\\
			\orr P(A) &= 0.486 \label{eq:P_A}	
		\end{align}
	\item
		\begin{align}
			\itt{Using Bayes' rule:}
			P(\bar{D}|H) &= \frac{P(H|\bar{D})P(\bar{D})}{P(H)} \label{eq:tmp3d1}\\
			\itt{Using \cref{eq:P_H} in \cref{eq:tmp3d1}:}
			\orr P(\bar{D}|H) &= \frac{(0.5)(0.7)}{0.62} \nonumber\\
			\orr P(\bar{D}|H) &\approx 0.5645 \label{eq:P_NOTDGivenH}
		\end{align}
	\item 
		\begin{align}
			\itt{Using Bayes' rule:}
			P(\bar{D}|A) &= \frac{P(A|\bar{D})P(\bar{D})}{P(A)} \label{eq:tmp3e1}\\
			\itt{Using Law of Total Probability to expand $P(A|\bar{D})$ :}
			\orr P(\bar{D}|A) &= \frac{\Big\{P(A|\bar{H}\bar{D})P(\bar{H}|\bar{D}) + P(A|\bar{D}H)P(H|\bar{D})\Big\}P(\bar{D})}{P(A)} \label{eq:tmp3e2}\\
			\itt{Using \cref{eq:P_A} in \cref{eq:tmp3e2}:}
			\orr P(\bar{D}|A) &= \frac{\Bigl\{(0.3)(0.5) + (0.6)(0.5)\Bigr\}(0.7)}{0.486} \nonumber\\
			\orr P(\bar{D}|A) &\approx 0.6481 \label{eq:P_NOTDGivenA}
		\end{align}
	\item 
		\begin{align}
			\itt{Check if:}
			 P(\{A+B\}|\bar{D}) &= P(\{A+B\})? \label{eq:toCheck3f}\\
			\itt{Let event $\text{$Pass$} = \set{\text{"Student obtained $A$ or $B$ grade."}$}}
			\itt{Or,} 
			Pass &= A + B \label{eq:tmp3f1}\\
			\itt{Starting from the LHS of \cref{eq:toCheck3f}, using \cref{eq:tmp3f1} and the Law of Total Probability:}
			P(Pass|\bar{D}) &= P(Pass|\bar{D}\bar{H})P(\bar{H}|\bar{D}) + P(Pass|\bar{D}H)P(H|\bar{D}) \nonumber\\
			\orr P(Pass|\bar{D}) &= (0.7)(0.5) + (1.0)(0.5) \nonumber\\
			\orr P(Pass|\bar{D}) &= 0.85 \label{eq:P_PassGivenNOTD}
			\itt{Now starting with the RHS of \cref{eq:toCheck3f} and using the Law of Total Probability:}
			P(Pass) &= P(Pass|\bar{D})P(\bar{D}) + P(Pass|D)P(D) \nonumber\\
			\orr P(Pass) &= P(Pass|\bar{D})P(\bar{D}) \nonumber\\
			{} &\quad+ \Bigl\{P(Pass|D\bar{H})P(\bar{H}|D) \nonumber\\  
			{} &\quad\quad+ P(Pass|DH)P(H|D)\Bigr\}P(D) \label{eq:tmp3f1}\\
			\itt{Using \cref{eq:P_PassGivenNOTD} in \cref{eq:tmp3f1}:}
			P(Pass) &= (0.85)(0.7) + \Bigl\{(0.7)(0.1) \nonumber\\
			{} &\quad+ (1.0)(0.9)\Bigr\}(0.3) \nonumber\\
			\orr P(Pass) &= 0.886 \label{eq:P_Pass}
			\itt{Since $P(Pass|\bar{D})$ in \cref{eq:P_PassGivenNOTD} and $P(Pass)$ in \cref{eq:P_Pass} are not equal, therefore the events of a student passing and the student being lazy are NOT independent events. \Neutrey} \nonumber
		\end{align}
	\item 
		\begin{align}
			\itt{Check if:}
			P(B|\bar{D}) &= P(B)? \label{eq:toCheck3g}\\
			\itt{Or,} 
			\itt{Starting from the LHS of \cref{eq:toCheck3g}, and using the Law of Total Probability:}
			P(B|\bar{D}) &= P(B|\bar{D}\bar{H})P(\bar{H}|\bar{D}) + P(B|\bar{D}H)P(H|\bar{D}) \nonumber\\
			\orr P(B|\bar{D}) &= (0.4)(0.5) + (0.4)(0.5) \nonumber\\
			\orr P(B\bar{D}) &= 0.4 \label{eq:P_BGivenNOTD}
			\itt{Now starting with the RHS of \cref{eq:toCheck3g} and using the Law of Total Probability:}
			P(B) &= P(B|\bar{H})P(\bar{H}) + P(B|H)P(H) \nonumber\\ \label{eq:tmp3f1}\\
			\orr P(B) &= (0.4)P(\bar{H}) + (0.4)P(H) \nonumber\\
			\orr P(B) &= 0.4 \label{eq:P_B}\\
			\itt{Since $P(B|\bar{D})$ in \cref{eq:P_BGivenNOTD} and $P(B)$ in \cref{eq:P_B} are equal, therefore the events of a student getting a $B$ grade and the student being lazy are independent events. \Smiley} \nonumber
		\end{align}
	\item
		\begin{align}
			\itt{Using Bayes' rule on $P(H|\{\bar{D}+A\})$:} 
			P(H|\{\bar{D}+A\}) &= \frac{P(\{\bar{D}+A\}|H)P(H)}{P(\{\bar{D}+A\})} \nonumber\\
			\orr P(H|\{\bar{D}+A\}) &= \frac{P(\{\bar{D}+DA\}|H)P(H)}{P(\{\bar{D}+DA\})} \nonumber\\
			\itt{Using Axiom 1 of Probability:}
			P(H|\{\bar{D}+A\}) &= \frac{\Bigl\{P(\bar{D}|H)+P(DA|H)\Bigr\}P(H)}{\{P(\bar{D})+P(DA)\}} \nonumber\\
			\orr P(H|\{\bar{D}+A\}) &= \frac{\Bigl\{P(\bar{D}|H)+\frac{P(DAH)}{P(H)}\Bigr\}P(H)}{\{P(\bar{D})+ P(DAH) + P(DA\bar{H})\}} \nonumber\\
			\orr P(H|\{\bar{D}+A\}) &= \frac{\Bigl\{P(\bar{D}|H)+\frac{P(DAH)}{P(H)}\Bigr\}P(H)}{\{P(\bar{D})+ P(DAH) + P(A|D\bar{H})P(\bar{H}|D)P(D)\}} \label{eq:tmp3h1}\\
			\itt{Using \cref{eq:P_NOTDGivenH,eq:P_H,eq:P_H,eq:P_DHA} in \cref{eq:tmp3h1}:}								P(H|\{\bar{D}+A\}) &= \frac{\Bigl\{0.5645+\frac{0.162}{0.62}\Bigr\}(0.62)}{\{0.7+ 0.162 + (0.3)(0.1)(0.3)\}} \nonumber\\
			\orr P(H|\{\bar{D}+A\}) &\approx 0.5878 \label{eq:P_HGivenNOTDORA}	
		\end{align}
\end{enumerate}

\noindent\rule{\textwidth}{1pt}