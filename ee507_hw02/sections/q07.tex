\section{Problem 7}
You throw a dart at a circular dartboard with unit radious, and have an equal probability of hitting each point on the dartboard. Let $R$ be the distance of the dart from the centre of the dartboard. Please answer the following questions.

\begin{enumerate}[7a.]
	\item Please argue that $R$ is a random variable.
	\item Please find the CDF of $R$.
	\item What is the probability that $0.2<R<0.5$ ?
	\item Please find the probability density function (pdf) of $R$.
	\item Now let $Q$ be the vertical distance of the dart from the bottom of the dartboard. Please find the pdf and CDF of Q.
\end{enumerate}
\subsection{Solution}

Refer to \cref{fig:dartBoardRandomVariable} for the calculations.

\begin{figure}[H]
	\centering
	\begin{adjustbox}{max width=0.9\textwidth}
		\import{../figures/}{dartBoardRandomVariable.pdf_tex}
	\end{adjustbox}
	\caption{Dartboard}
	\label{fig:dartBoardRandomVariable}
\end{figure}

\begin{enumerate}[7a.]
	\item Given that $R$ is the distance from the centre of the dartboard it can be used to represent every possible experiment outcome with a number. In other words, $R$ is an injective (one-one) mapping from the set of all of outcomes to a set of numbers. Also $R$ as a random variable follows a set of trivial rules accounting for impossible events, via $P_R(r\rightarrow-\infty) = 0$ (a dart cannot be thrown at a location at a negative distance from the centre) and $P_R(r\rightarrow\infty) = 0$ (possibility of dart being thrown out of the dart board has been rules out as per the problem statement).
	
	\item CDF$_R(\alpha) = P_R(r<\alpha)$ is simply the area of the circle whose centre coincides with the dart board's centre and whose radius is R.\\[5pt]
	$\text{CDF}_R(\alpha) = \threepartdef{0}{\alpha < 0}{\frac{\pi \alpha^2}{\pi 1^2} = \alpha^2}{\alpha \in [0,1)}{1}{\alpha \geq 1}$
	
	\item 
		\begin{align}
			P_R(r \in (0.2, 0.5)) &= \text{CDF}_R(0.5) - \text{CDF}_R(0.2) \nonumber\\
			\orr P_R(r \in (0.2, 0.5)) &= \pi0.25 - \pi0.04 \nonumber\\
			\orr P_R(r \in (0.2, 0.5)) &\approx  0.6597 \nonumber
		\end{align}
	
	\item pdf is the derivative of CDF wrt the random variable.\\
	$f_R(\alpha) = \frac{dF_R(\alpha)}{d\alpha} = \threepartdef{0}{\alpha < 0}{2\alpha}{\alpha \in [0,1)}{0}{\alpha \geq 1}$
\end{enumerate}	

\noindent\rule{\textwidth}{1pt}
