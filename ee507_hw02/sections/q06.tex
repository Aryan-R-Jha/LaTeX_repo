\section{Problem 6}
A simple dartboard is shown below:

\begin{figure}[!ht]
	\centering
	\import{../figures/}{dartBoardPdf.pdf_tex}
	\caption{A dart board with given scores for hitting different regions.}
	\label{fig:dartBoard}
\end{figure}

Your favourite EE 507 instructor throws a dart at the board. He hits the board with probability $0.7$ and misses the board with probability $0.3$. Given that he hit the board, he is equally likely to hit any point on it. He receives a score of $0$ if he misses the board, and receives the score shown if he hits the board.

\begin{enumerate}[a.]
	\item What are the outcomes of this experiment? How many are there?
	\item What are the events in this experiment? How many are there?
	\item What is the probability that your instructor's score is more than $3$?
	\item Let's say we repeated the experiment twice, independently. What is the probability that the score is greater than $3$ on the first try, and less than or equal to 3 on the second try?\\
	Also what is the probability that the two scores are different?
\end{enumerate}

\subsection{Solution}

Let the experiment of the instructor throwing the dart on the board be called $E$.
\begin{enumerate}[a.]
	\item The outcomes can be listed as $\Omega = \set{A_0, A_2, A_4, A_6, A_8}$ where $A_i$ refers to scoring $i$ points with the dart. $A_0$ represents the dart missing the board and scoring zero points. \\
	In total there are $5$ outcomes for this experiment.
	\item An event is a subset of the power set of the set of outcomes ($\Omega$) of the experiment.\\
	The power set of $\Omega$ is $\powerset$: 
	\begin{align}
		\powerset{} &= \{\set\phi, \set{A_0}, \set{ A_2}, \cdots  \set{A_8}, \nonumber\\
		& \set{A_0, A_2}, \set{A_0, A_4} \cdots \set{A_6, A_8}, \nonumber \\ 
		& \cdots \set{A_0, A_2, A_4, A_6, A_8}\} \nonumber 
	\end{align}
	Say, a subset $\set{A_0, A_4}$ of $\powerset$ represents the event \set{The dart misses the board OR The dart hits the board in the region with $4$ points}.\\
	There are $2^5 = 32$ events for this experiment.
	\item $P(\text{Score} > 3)$ = $P(\set{A_4, A_6, A_8})$.\\ Outcomes are always disjoint events, so we can use Axiom 3. \\
	$P(\set{A_4, A_6, A_8})$ = $P(A_4) + P(A_6) + P(A_8)$ = $3*0.7/4 = 0.525$.
	\item Repeated trials are independent trials. We can simply multiply the individual probabilities to obtain the final probability. \\
	$P(\text{Score} > 3 \text{ on the first trial AND Score} < 3 \text{ on the second trials}) = P(\text{Score} > 3)*P(\text{Score} \leq 3) = 0.525(1-0.525) = 0.249375$\\[10pt]
	$P(\text{Both Scores Different}) = 1 - P(\text{Both Scores Same}) \\
	= 1 - P(\set{\set{A_0 A_0}, \set{A_2 A_2}, \set{A_4 A_4}, \set{A_6 A_6}, \set{A_8 A_8}}) \\
	= 1 - [0.3^2 + 4*\{\frac{0.7}{4}\}^2] = 0.7875$
\end{enumerate}

\noindent\rule{\textwidth}{1pt}