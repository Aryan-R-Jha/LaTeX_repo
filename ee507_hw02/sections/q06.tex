\section{Problem 6}
An experiment has four outcomes $A, B, C $ and $D$ which have probabilities $P(A)=0.4$, $P(B)=0.3$, $P(C)=0.2$, and $P(D)=0.1$. We define a random varialbe $x$ as follows: $X(A) = 1$, $X(B) = 5$, $X(C) = 2$, $X(D) = 1$. Also, we define the event $Z = \set{A, B, C}$. Please answer the following questions:

\begin{enumerate}[6a.]
	\item Find and plot the CDF of $X$.
	\item What is the probability that $2<X<6$?
	\item Find and plot the probability mass function (pmf) of $X$.
	\item Find the CDF of $X$ given $Z$ has occurred.
	\item What is the probability of $Z$, given that $X>1.5$?
\end{enumerate}

\subsection{Solution}
Refer to \cref{tab:pmf6a1} for the calculations.

\begin{enumerate}[6a.]
	\item 
		\begin{table}[H]
			\centering
			\caption{Random Variable vs pmf and CDF values for the experiment.}
			\label{tab:pmf6a1}
			\begin{tabular}{cccc}
				\toprule
				$X$ & \text{outcomes} & $\text{pmf}_X(\alpha)$ & $F_X(\alpha)$\\
				\midrule
				$1$ & $A, D$  & $0.5$ & $0.5$ \\
				$2$ & $C$     & $0.2$ & $0.7$ \\
				$5$ & $B$     & $0.3$ & $1.0$ \\
				\bottomrule
			\end{tabular}
		\end{table}
	
		\begin{equation}
			\text{CDF}_X(\alpha) = F_X(\alpha) = \fourpartdef{0}{\alpha<1}{0.5}{\alpha \in [1,2)}{0.7}{\alpha \in [2,5)}{1.0}{\alpha \geq 5} \nonumber
		\end{equation}
		
		\begin{figure}[H]
			\centering
			\begin{adjustbox}{max width=\textwidth}
				\import{../figures/}{cdf1.pdf_tex}
			\end{adjustbox}
			\caption{CDF plot of the experiment.}
			\label{fig:cdf6a1}
		\end{figure}
	
	\item 
		\begin{align}
			P_X(x \in (2,6)) &= F_X(x \rightarrow 6^-) - F_X(x \rightarrow 2^+) \nonumber\\
			\orr P_X(x \in (2,6)) &= 1.0 - 0.7 \nonumber\\
			\orr P_X(x \in (2,6)) &= 0.3 \nonumber
		\end{align}
	
	\item Refer to \cref{tab:pmf6c1} for the values.
		\begin{table}[H]
			\centering
			\caption{Probability Mass Function (pmf) of the experiment.}
			\label{tab:pmf6c1}
			\begin{tabular}{cc}
				\toprule
				$X$ & PMF$_X(x)$ \\
				\midrule
				$1$ & $0.5$ \\
				$2$ & $0.2$ \\
				$3$ & $0.3$ \\
				\bottomrule
			\end{tabular}
		\end{table}
	
	\item 
		To compute: $F_{X|Z}(x|z)$\\
		But $Z = \bar{D}$. \\
		So we can compute the CDF for outcomes $A,B$ and $C$ given that $D$ has NOT occurred. The same has been done in \cref{tab:pmf6d1}.
		
		\begin{table}[H]
			\centering
			\caption{Modified pmf and CDF values of the experiment given that event $Z$ has occurred.}
			\label{tab:pmf6d1}
			\begin{tabular}{cccc}
				\toprule
				$X_Z$ & outcome & pmf$_{X|Z}(\alpha)$ & $F_{X|Z}(\alpha)$ \\
				\midrule
				$1$ & $A$ & $\frac{0.4}{1-0.1} = \frac{4}{9}$ & $\frac{4}{9}$ \\[5pt]
				$2$ & $C$ & $\frac{0.2}{1-0.1} = \frac{2}{9}$ & $\frac{6}{9}$ \\[5pt]
				$5$ & $B$ & $\frac{0.3}{1-0.1} = \frac{3}{9}$ & $1$ \\
				\bottomrule
			\end{tabular}
		\end{table}
	
		\begin{equation}
			\text{CDF}_{X|Z}(\alpha) = F_{X|Z}(\alpha) = \fourpartdef{0}{\alpha<1}{\frac{4}{9}}{\alpha \in [1,2)}{\frac{6}{9}}{\alpha \in [2,5)}{1.0}{\alpha \geq 5} \nonumber
		\end{equation} 
		
		\begin{figure}[H]
			\centering
			\begin{adjustbox}{max width=\textwidth}
				\import{../figures/}{cdf2.pdf_tex}
			\end{adjustbox}
			\caption{Modified CDF plot of the experiment given that event $Z$ has occurred.}
			\label{fig:cdf6d1}
		\end{figure}
	
	\item
		\begin{align}
			P(Z|X>1.5) &= \frac{P(X>1.5|Z)P(Z)}{P(X>1.5)} \nonumber\\
			\orr P(Z|X>1.5) &= \frac{(1-F_{X|Z}(1.5))(0.9)}{1-F_X(1.5)} \nonumber\\
			\orr P(Z|X>1.5) &= \frac{\Bigl\{1-\frac{4}{9}\Bigr\}(0.9)}{1-0.5} \nonumber\\
			\orr P(Z|X>1.5) &= 1 \nonumber\\
			\itt{ \Smiley Of course.. $\set{X>1.5} = \set{B, C} \in Z$}. \nonumber
		\end{align}
\end{enumerate}

\noindent\rule{\textwidth}{1pt}