\chapter{Models: Battery Model for Mutli-Period OPF}

% define T, G, B, \Delta t etc.

\begin{table}[htbp]
	\label{tab:grid_Nazir2018Jun}
	\centering
	\caption{Description of Grid Parameters}
	\begin{tabular}{>{\raggedright\arraybackslash $}p{2.5cm}<{$} 
		>{\raggedright\arraybackslash}p{5cm}}
		\toprule
		\text{Variable} & \text{Description}\\
		\midrule
		\mathcal{G} & Set of all nodes \\
		\mathcal{L} & {} \\
		\mathcal{T} & {} \\
		\mathcal{\Delta t} & {} \\
		\bottomrule
	\end{tabular}%
\end{table}%

In \cite{Nazir2018Jun, Nazir2019Jun}, the batteries are modelled using four state/control variables, which are:
\begin{table}[htbp]
	\label{tab:batt_Nazir2018Jun}
	\centering
	\caption{Description of Battery Variables}
	\begin{tabular}{>{\raggedright\arraybackslash $}p{2.5cm}<{$} 
		>{\raggedright\arraybackslash}p{5cm} 
		>{\centering\arraybackslash $}p{2.5cm}<{$} 
		>{\centering\arraybackslash\arraybackslash $}p{2.5cm}<{$}}
		\toprule
		\text{Variable} & \text{Description} & \text{Dimension} & \text{Dimension $pu$} \\
		\midrule
		B_{n, k} & State of Charge (SOC) of Battery& [kWh] & [pu\,h] \\
		P^c_{n, k} & Average Charging Power of the Battery during the $k$-th time interval. & [kW] & [pu] \\
		P^d_{n, k} & Average Discharging Power of the Battery during the $k$-th time interval. & [kW] & [pu] \\
		q_{B_{n, k}} & Average Reactive Power Output from the Battery Inverter during the $k$-th time interval. & [kVAr] & [pu] \\
		\bottomrule
	\end{tabular}%
\end{table}%

\clearpage

\begin{description}
	\item[$k \in \{0, 1, 2, \ldots, T/\Delta t\}$] The index of the discretized time intervals, where $k$ represents
	the $k$-th time interval of duration $\Delta t$ within the continuous time range
	$[0, T]$. The corresponding continuous time interval is $[(k-1)\Delta t, k\Delta t]$.
	\item[$n \in \mathcal{G}$] The node $n$ is an element of the set of all nodes
	in the power grid $\mathcal{G}$.
	% Add more items as needed
\end{description}


