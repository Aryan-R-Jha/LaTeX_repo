\chapter{Models: Battery Model for Mutli-Period OPF}

\begin{table}[htbp]
	\label{tab:grid_Nazir2018Jun}
	\centering
	\caption{Description of Grid Parameters}
	\begin{tabular}{>{\raggedright\arraybackslash $}p{2.5cm}<{$} 
		>{\raggedright\arraybackslash}p{7.5cm}}
		\toprule
		\text{Variable} & \text{Description}\\
		\midrule
		\mathcal{N} & {Set of all nodes. $\mathcal{N} = \{1,2, \ldots n\}$} \\
		\mathcal{L} & {Set of all branches. $\mathcal{L} = 
		\{1,2, \ldots l\} = \{(i, k)\} \subset (\mathcal{N} \times \mathcal{N})$.} \\
		\mathcal{D} & {Set of all nodes containing DERs. $\mathcal{D} 
		\subset \mathcal{N}$} \\
		\mathcal{B} & {Set of all nodes containing storage. $\mathcal{B} 
		\subset \mathcal{N}$} \\
		{\Delta t} & {Duration of a single time period. Here $\Delta t = 15 
		\text{ min.} = 0.25 \text{ h}$.} \\
		T & {Prediction Horizon Duration. Total time duration solved for as part
		of one instance of MP-OPF.} \\
		K & {Prediction Horizon Number. Total number of discrete-time steps 
		solved for in one instance of MP-OPF. $T = K \Delta t$} \\
		\bottomrule
	\end{tabular}%
\end{table}%

In \cite{Nazir2018Jun, Nazir2019Jun}, the batteries are modelled using 
four state/control variables, which are:
\begin{table}[htbp]
	\label{tab:batt_Nazir2018Jun}
	\centering
	\caption{Description of Battery Variables}
	\begin{tabular}{>{\raggedright\arraybackslash $}p{2.5cm}<{$} 
		>{\raggedright\arraybackslash}p{5cm} 
		>{\centering\arraybackslash $}p{2.5cm}<{$} 
		>{\centering\arraybackslash\arraybackslash $}p{2.5cm}<{$}}
		\toprule
		\text{Variable} & \text{Description} & \text{Dimension} & 
		\text{Dimension $pu$} \\
		\midrule
		B_{n, k} & State of Charge (SOC) of Battery& [kWh] & [pu\,h] \\
		P^c_{n, k} & Average Charging Power of the Battery during the 
		$k$-th time interval. & [kW] & [pu] \\
		P^d_{n, k} & Average Discharging Power of the Battery during the 
		$k$-th time interval. & [kW] & [pu] \\
		q_{B_{n, k}} & Average Reactive Power Output from the Battery 
		Inverter during the $k$-th time interval. & [kVAr] & [pu] \\
		\bottomrule
	\end{tabular}%
\end{table}%

\clearpage
\text{where, }
\begin{description}
	\item[$k \in \{0, 1, 2, \ldots, T/\Delta t\}$] The index of the discretized time intervals, where $k$ represents
	the $k$-th time interval of duration $\Delta t$ within the continuous time range
	$[0, T]$. The corresponding continuous time interval is $[(k-1)\Delta t, k\Delta t]$.
	\item[$n \in \mathcal{N}$] The node $n$ is an element of the set of all nodes
	in the power grid $\mathcal{N}$. Note that $n$ can be used both as an iterator or
	as the total number of nodes in the grid (i.e. the cardinality of $\mathcal{N}$),
	and its meaning should be obvious from context.
	% Add more items as needed
\end{description}

\begin{table}[htbp]
	\label{tab:bounds_batt_Nazir2018Jun}
	\centering
	\caption{Values, Lower Bounds and Upper Bounds on Battery Variables}
	\begin{tabular}{>{\raggedright\arraybackslash $}p{2.5cm}<{$}
		>{\raggedright\arraybackslash $}p{5.5cm}<{$}
		>{\raggedright\arraybackslash}p{5.5cm}<{}}
		\toprule
		\text{Variable} & \text{Value or Limits} & \text{Description}\\
		\midrule
		{P_{Max}} & {P_{Rated} \text{ of corresponding DER.}} & {}\\
		{q_{B_{Max}}} & {q_{D_{Rated}} \text{ of corresponding DER.}} & {}\\
		{P_d, P_c} & {[0, P_{Max}]} & {}\\
		{E_{Rated}} & {P_{Max} \times 4\text{ h}} & {$4 \text{ h}$ of 
		one-way Charging/Discharging at Maximum Power}\\
		{B} & {[0.30E_{Rated}, 0.95E_{Rated}]} & {$2.4 \text{ h}$ of
		one-way Charging/Discharging at Maximum Power}\\
		{B_0} & {0.625E_{Rated}} & {Batteries start with an SOC value in
		the middle of their SOC range.} \\
		{\eta_d , \eta_c} & {0.8} & {} \\
		\bottomrule
	\end{tabular}%
\end{table}%
% 